\documentclass[12pt,a4paper]{article}
\usepackage[utf8]{inputenc}
\usepackage[T1]{fontenc}
\usepackage[indonesian]{babel}
\usepackage{geometry}
\usepackage{longtable}
\usepackage{array}
\usepackage{booktabs}
\usepackage{enumitem}
\usepackage[hidelinks,breaklinks]{hyperref}
\geometry{margin=2.5cm}

\begin{document}

\begin{center}
\textbf{\Large RENCANA PEMBELAJARAN SEMESTER (RPS)}\\
\textbf{Berbasis Outcome-Based Education (OBE)}\\[0.5cm]
\textbf{PROGRAM STUDI TEKNIK INFORMATIKA}\\
\textbf{FAKULTAS TEKNIK}\\
\textbf{UNIVERSITAS LOREM IPSUM}\\
\end{center}

\vspace{1cm}

% ============================================================
% 1. IDENTITAS MATA KULIAH
% ============================================================
\section*{1. Identitas Mata Kuliah}
\begin{tabular}{ll}
Nama Program Studi & : Teknik Informatika \\
Nama Mata Kuliah & : Pemrograman Berorientasi Objek \\
Kode Mata Kuliah & : TIF-301 \\
Semester & : 3 (Tiga) \\
SKS / Bobot Kredit & : 3 SKS (2 Teori, 1 Praktikum) \\
Dosen Pengampu & : Dr. Lorem Ipsum, M.Kom. \\
Tanggal Penyusunan & : 31 Januari 2026 \\
\end{tabular}

\vspace{0.5cm}

% ============================================================
% 2. CAPAIAN PEMBELAJARAN LULUSAN (CPL)
% ============================================================
\section*{2. Capaian Pembelajaran Lulusan (CPL)}

CPL yang dibebankan pada mata kuliah ini mencakup kompetensi lulusan dalam aspek pengetahuan, keterampilan, dan sikap:

\begin{itemize}[leftmargin=*]
  \item \textbf{CPL-1 (Pengetahuan):} Menguasai konsep teoretis bidang pengetahuan tertentu secara umum dan konsep teoretis bagian khusus dalam bidang pengetahuan tersebut secara mendalam, serta mampu memformulasikan penyelesaian masalah prosedural.
  
  \item \textbf{CPL-2 (Keterampilan Umum):} Mampu menerapkan pemikiran logis, kritis, sistematis, dan inovatif dalam konteks pengembangan atau implementasi ilmu pengetahuan dan teknologi yang memperhatikan dan menerapkan nilai humaniora.
  
  \item \textbf{CPL-3 (Keterampilan Khusus):} Mampu merancang, mengimplementasikan, dan mengevaluasi sistem perangkat lunak menggunakan paradigma berorientasi objek dengan mempertimbangkan standar kualitas perangkat lunak.
  
  \item \textbf{CPL-4 (Sikap):} Menunjukkan sikap bertanggung jawab atas pekerjaan di bidang keahliannya secara mandiri dan mampu bekerja sama dalam tim.
\end{itemize}

\vspace{0.5cm}

% ============================================================
% 3. CAPAIAN PEMBELAJARAN MATA KULIAH (CPMK)
% ============================================================
\section*{3. Capaian Pembelajaran Mata Kuliah (CPMK)}

Kemampuan atau kompetensi spesifik yang diharapkan mahasiswa kuasai setelah menyelesaikan mata kuliah:

\begin{itemize}[leftmargin=*]
  \item \textbf{CPMK-1:} Mahasiswa mampu memahami dan menjelaskan konsep dasar pemrograman berorientasi objek (class, object, encapsulation, inheritance, polymorphism).
  
  \item \textbf{CPMK-2:} Mahasiswa mampu merancang solusi permasalahan menggunakan diagram UML (Use Case, Class Diagram, Sequence Diagram).
  
  \item \textbf{CPMK-3:} Mahasiswa mampu mengimplementasikan konsep OOP dalam bahasa pemrograman Java dengan menerapkan prinsip SOLID.
  
  \item \textbf{CPMK-4:} Mahasiswa mampu menganalisis dan mengevaluasi kualitas kode program berorientasi objek berdasarkan best practices dan design patterns.
\end{itemize}

\vspace{0.5cm}

% ============================================================
% 4. SUB-CPMK / INDIKATOR PENCAPAIAN
% ============================================================
\section*{4. Sub-CPMK / Indikator Pencapaian}

Penjabaran CPMK menjadi indikator yang lebih terukur dan dapat diuji:

\begin{itemize}[leftmargin=*]
  \item \textbf{Sub-CPMK 1.1:} Menjelaskan perbedaan antara pemrograman prosedural dan berorientasi objek
  \item \textbf{Sub-CPMK 1.2:} Mengidentifikasi class, object, dan attribute dalam studi kasus nyata
  \item \textbf{Sub-CPMK 1.3:} Mendemonstrasikan konsep encapsulation dengan access modifier
  \item \textbf{Sub-CPMK 2.1:} Membuat use case diagram untuk sistem sederhana
  \item \textbf{Sub-CPMK 2.2:} Merancang class diagram dengan relasi yang tepat (association, aggregation, composition)
  \item \textbf{Sub-CPMK 3.1:} Mengimplementasikan inheritance dan method overriding
  \item \textbf{Sub-CPMK 3.2:} Menerapkan polymorphism dalam desain aplikasi
  \item \textbf{Sub-CPMK 4.1:} Menganalisis code smell dan melakukan refactoring
  \item \textbf{Sub-CPMK 4.2:} Mengimplementasikan minimal 3 design patterns (Singleton, Factory, Observer)
\end{itemize}

\vspace{0.5cm}

% ============================================================
% 5. MATERI PEMBELAJARAN (BAHAN KAJIAN)
% ============================================================
\section*{5. Materi Pembelajaran (Bahan Kajian)}

Daftar topik materi yang relevan dengan Sub-CPMK dan CPMK:

\begin{enumerate}[leftmargin=*]
  \item Pengenalan Paradigma Berorientasi Objek
  \item Konsep Class dan Object
  \item Encapsulation dan Information Hiding
  \item Inheritance dan Hierarchical Relationships
  \item Polymorphism dan Dynamic Binding
  \item Abstract Class dan Interface
  \item UML Diagrams (Use Case, Class, Sequence)
  \item Exception Handling dalam OOP
  \item Prinsip SOLID (Single Responsibility, Open/Closed, Liskov Substitution, Interface Segregation, Dependency Inversion)
  \item Design Patterns (Creational, Structural, Behavioral)
  \item Collections dan Generics
  \item File I/O dan Serialization
  \item Unit Testing dan Test-Driven Development
  \item Code Quality dan Refactoring
\end{enumerate}

\vspace{0.5cm}

% ============================================================
% 6. METODE PEMBELAJARAN
% ============================================================
\section*{6. Metode Pembelajaran}

Strategi atau pendekatan pembelajaran yang dipilih sesuai OBE yang menekankan aktivitas mahasiswa:

\begin{itemize}[leftmargin=*]
  \item \textbf{Ceramah Interaktif:} Penjelasan konsep dengan diskusi tanya jawab
  \item \textbf{Problem-Based Learning (PBL):} Mahasiswa menyelesaikan permasalahan nyata menggunakan OOP
  \item \textbf{Project-Based Learning:} Pengembangan aplikasi mini sebagai proyek kelompok
  \item \textbf{Praktikum Terbimbing:} Latihan coding dengan bimbingan asisten
  \item \textbf{Peer Review:} Mahasiswa melakukan code review terhadap pekerjaan rekan
  \item \textbf{Flipped Classroom:} Mahasiswa mempelajari materi sebelum kelas, diskusi mendalam di kelas
  \item \textbf{Studi Kasus:} Analisis implementasi OOP pada aplikasi open source
\end{itemize}

\vspace{0.5cm}

% ============================================================
% 7. PENGALAMAN BELAJAR MAHASISWA
% ============================================================
\section*{7. Pengalaman Belajar Mahasiswa}

Deskripsi tugas, aktivitas, atau pengalaman belajar yang mendukung pencapaian Sub-CPMK:

\begin{itemize}[leftmargin=*]
  \item Menganalisis studi kasus untuk mengidentifikasi class dan object
  \item Merancang class diagram untuk sistem perpustakaan, e-commerce, atau sistem akademik
  \item Mengimplementasikan aplikasi console berbasis OOP dengan minimal 5 class
  \item Melakukan refactoring pada kode yang diberikan untuk meningkatkan kualitas
  \item Berkolaborasi dalam tim untuk mengembangkan proyek aplikasi desktop sederhana
  \item Mempresentasikan hasil analisis design pattern pada framework populer
  \item Menulis unit test untuk class yang telah dibuat
  \item Melakukan peer code review dan memberikan feedback konstruktif
  \item Membuat dokumentasi teknis menggunakan Javadoc
\end{itemize}

\vspace{0.5cm}

% ============================================================
% 8. KRITERIA, INDIKATOR, DAN BOBOT PENILAIAN
% ============================================================
\section*{8. Kriteria, Indikator, dan Bobot Penilaian}

Teknik/alat asesmen dipetakan ke Sub-CPMK/CPMK dengan bobot yang jelas:

\begin{longtable}{|>{\raggedright\arraybackslash}p{2.5cm}|>{\raggedright\arraybackslash}p{4cm}|>{\raggedright\arraybackslash}p{5.5cm}|c|}
\hline
\textbf{Komponen} & \textbf{Teknik Asesmen} & \textbf{Indikator/CPMK} & \textbf{Bobot (\%)} \\
\hline
\endfirsthead
\hline
\textbf{Komponen} & \textbf{Teknik Asesmen} & \textbf{Indikator/CPMK} & \textbf{Bobot (\%)} \\
\hline
\endhead
\hline
\endfoot

Tugas Individu & Coding Assignment & Sub-CPMK 1.1, 1.2, 1.3, 3.1 & 15 \\
\hline
Kuis & Multiple Choice \& Essay & Sub-CPMK 1.1, 2.1, 2.2 & 10 \\
\hline
Praktikum & Lab Exercise & Sub-CPMK 3.1, 3.2, 4.1 & 15 \\
\hline
UTS & Written Exam \& Coding & CPMK-1, CPMK-2 & 20 \\
\hline
Proyek Kelompok & Application Development & CPMK-2, CPMK-3, CPMK-4 & 20 \\
\hline
UAS & Comprehensive Exam & CPMK-1, CPMK-2, CPMK-3, CPMK-4 & 20 \\
\hline
\textbf{Total} & & & \textbf{100} \\
\hline
\end{longtable}

\textbf{Kriteria Penilaian:}
\begin{itemize}[leftmargin=*]
  \item A (85-100): Menguasai semua CPMK dengan sangat baik, mampu menerapkan dalam kasus kompleks
  \item B (70-84): Menguasai sebagian besar CPMK dengan baik
  \item C (60-69): Menguasai CPMK dasar dengan cukup
  \item D (50-59): Menguasai sebagian kecil CPMK
  \item E (<50): Belum menguasai CPMK yang ditetapkan
\end{itemize}

\vspace{0.5cm}

% ============================================================
% 9. EVALUASI DAN REFLEKSI PEMBELAJARAN (OPSIONAL)
% ============================================================
\section*{9. Evaluasi dan Refleksi Pembelajaran}

Penilaian sumatif/formatif untuk memantau ketercapaian outcome secara menyeluruh:

\begin{itemize}[leftmargin=*]
  \item \textbf{Evaluasi Formatif:} Kuis mingguan, latihan coding, peer review untuk memberikan feedback berkelanjutan
  \item \textbf{Evaluasi Sumatif:} UTS dan UAS untuk mengukur pencapaian CPMK secara komprehensif
  \item \textbf{Refleksi Mahasiswa:} Jurnal belajar mingguan untuk refleksi diri terhadap pemahaman materi
  \item \textbf{Evaluasi Dosen:} Survey kepuasan mahasiswa di tengah dan akhir semester
  \item \textbf{Continuous Improvement:} Analisis hasil penilaian untuk perbaikan RPS di semester berikutnya
\end{itemize}

\vspace{0.5cm}

% ============================================================
% 10. DAFTAR REFERENSI
% ============================================================
\section*{10. Daftar Referensi}

Sumber belajar utama yang digunakan dalam penyusunan materi dan asesmen:

\begin{enumerate}[leftmargin=*]
  \item Gamma, E., Helm, R., Johnson, R., \& Vlissides, J. (1994). \textit{Design Patterns: Elements of Reusable Object-Oriented Software}. Addison-Wesley.
  
  \item Bloch, J. (2018). \textit{Effective Java} (3rd ed.). Addison-Wesley Professional.
  
  \item Martin, R. C. (2017). \textit{Clean Architecture: A Craftsman's Guide to Software Structure and Design}. Prentice Hall.
  
  \item Horstmann, C. S. (2019). \textit{Core Java Volume I--Fundamentals} (11th ed.). Prentice Hall.
  
  \item Freeman, E., \& Robson, E. (2020). \textit{Head First Design Patterns} (2nd ed.). O'Reilly Media.
  
  \item Fowler, M. (2018). \textit{Refactoring: Improving the Design of Existing Code} (2nd ed.). Addison-Wesley Professional.
  
  \item Oracle. (2024). \textit{The Java Tutorials}. Retrieved from \url{https://docs.oracle.com/javase/tutorial/}
  
  \item Larman, C. (2004). \textit{Applying UML and Patterns: An Introduction to Object-Oriented Analysis and Design} (3rd ed.). Prentice Hall.
\end{enumerate}

\vspace{1cm}

\begin{flushright}
\begin{tabular}{c}
Disusun oleh,\\[2cm]
\textbf{Dr. Lorem Ipsum, M.Kom.}\\
NIP. 123456789012345678
\end{tabular}
\end{flushright}

\end{document}
