\documentclass[12pt,a4paper]{article}
\usepackage[utf8]{inputenc}
\usepackage[T1]{fontenc}
\usepackage[indonesian]{babel}
\usepackage{geometry}
\usepackage{longtable}
\usepackage{array}
\usepackage{booktabs}
\geometry{margin=2.5cm}

\begin{document}

\begin{center}
\textbf{\Large RENCANA PEMBELAJARAN SEMESTER (RPS)}\\
\textbf{PROGRAM STUDI ...}\\
\textbf{FAKULTAS ...}\\
\end{center}

\section*{A. Identitas Mata Kuliah}
\begin{tabular}{ll}
Nama Mata Kuliah & : ... \\
Kode Mata Kuliah & : ... \\
Rumpun Mata Kuliah & : ... \\
Bobot (SKS) & : ... \\
Semester & : ... \\
Dosen Pengampu & : ... \\
Tanggal Penyusunan & : ... \\
\end{tabular}

\section*{B. Capaian Pembelajaran Lulusan (CPL)}
\begin{itemize}
  \item CPL-1: ...
  \item CPL-2: ...
\end{itemize}

\section*{C. Capaian Pembelajaran Mata Kuliah (CPMK)}
\begin{itemize}
  \item CPMK-1: ...
  \item CPMK-2: ...
\end{itemize}

\section*{D. Matriks Keterkaitan CPL dan CPMK}
\begin{tabular}{c|c|c}
\toprule
CPL & CPMK & Tingkat Kontribusi \\
\midrule
CPL-1 & CPMK-1 & Tinggi \\
CPL-2 & CPMK-2 & Sedang \\
\bottomrule
\end{tabular}

\section*{E. Rencana Pembelajaran Semester}

\begin{longtable}{|c|p{3cm}|p{3cm}|p{3cm}|p{2.5cm}|p{2cm}|}
\hline
Minggu &
Sub-CPMK &
Materi Pembelajaran &
Metode Pembelajaran &
Penilaian &
Bobot (\%) \\
\hline
1 & ... & ... & ... & ... & ... \\
2 & ... & ... & ... & ... & ... \\
3 & ... & ... & ... & ... & ... \\
\hline
\end{longtable}

\section*{F. Strategi Pembelajaran}
Metode: ceramah, diskusi, PBL, project-based learning, dll.

\section*{G. Media dan Bahan Ajar}
Buku teks, modul, LMS, software, dll.

\section*{H. Sistem Penilaian}

\begin{tabular}{l c}
\toprule
Komponen & Bobot (\%) \\
\midrule
Tugas & ... \\
UTS & ... \\
UAS & ... \\
Proyek & ... \\
\bottomrule
\end{tabular}

\section*{I. Referensi}
\begin{enumerate}
  \item ...
  \item ...
\end{enumerate}

\end{document}
