\section{Peta Konsep Pemrograman Berorientasi Objek}

Mata kuliah ini mencakup 14 topik utama yang saling terkait:

\begin{enumerate}
  \item \textbf{Bab II}: Landasan Teori - Fondasi paradigma OOP
  \item \textbf{Bab III}: Class dan Object - Building blocks OOP
  \item \textbf{Bab IV}: Encapsulation - Information hiding
  \item \textbf{Bab V}: Inheritance - Code reuse dan hierarki
  \item \textbf{Bab VI}: Polymorphism - Fleksibilitas kode
  \item \textbf{Bab VII}: Abstract Class \& Interface - Abstraksi
  \item \textbf{Bab VIII}: UML Diagrams - Perancangan visual
  \item \textbf{Bab IX}: Exception Handling - Error management
  \item \textbf{Bab X}: Prinsip SOLID - Design principles
  \item \textbf{Bab XI}: Design Patterns - Solusi proven
  \item \textbf{Bab XII}: Collections, Generics, \& File I/O - Struktur data dan persistensi
  \item \textbf{Bab XIII}: Unit Testing, TDD, \& Refactoring - Quality assurance
  \item \textbf{Bab XIV}: Evaluasi Komprehensif - Integrasi semua konsep
\end{enumerate}

\textbf{Alur Pembelajaran:}
\begin{itemize}
  \item Bab II-IV: Konsep fundamental OOP
  \item Bab V-VII: Konsep lanjut OOP
  \item Bab VIII: Perancangan sistem
  \item Bab IX-XIII: Praktik baik, kualitas, dan tooling
  \item Bab XIV: Evaluasi dan integrasi
\end{itemize}
