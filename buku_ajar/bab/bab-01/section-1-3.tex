\section{Petunjuk Penggunaan Buku Ajar}

\subsection{Untuk Mahasiswa}

\textbf{Sebelum Perkuliahan:}
\begin{enumerate}
  \item Baca Sub-CPMK di awal bab untuk memahami target pembelajaran
  \item Pelajari materi pokok dengan seksama
  \item Jalankan dan modifikasi semua contoh kode yang diberikan
  \item Catat pertanyaan atau konsep yang belum dipahami
\end{enumerate}

\textbf{Selama Perkuliahan:}
\begin{enumerate}
  \item Diskusikan konsep yang sulit dengan dosen dan teman
  \item Kerjakan aktivitas pembelajaran secara aktif
  \item Tanyakan hal-hal yang belum jelas
  \item Berpartisipasi dalam code review dan diskusi kelompok
\end{enumerate}

\textbf{Setelah Perkuliahan:}
\begin{enumerate}
  \item Kerjakan latihan dan refleksi
  \item Lakukan asesmen mandiri
  \item Centang checklist kompetensi yang telah dikuasai
  \item Kerjakan proyek mini untuk memperdalam pemahaman
\end{enumerate}

\subsection{Untuk Dosen}

Buku ini dapat digunakan sebagai:
\begin{itemize}
  \item Bahan ajar utama untuk perkuliahan
  \item Sumber latihan dan tugas
  \item Referensi untuk menyusun soal ujian
  \item Panduan untuk merancang aktivitas pembelajaran
  \item Alat untuk mengukur pencapaian CPMK mahasiswa
\end{itemize}
