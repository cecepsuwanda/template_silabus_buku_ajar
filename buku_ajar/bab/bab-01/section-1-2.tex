\section[Keterkaitan Buku Ajar dengan RPS Berbasis OBE]{Keterkaitan Buku Ajar dengan RPS \protect\\ Berbasis OBE}

Buku ajar ini dirancang selaras dengan Rencana Pembelajaran Semester (RPS) mata kuliah Pemrograman Berorientasi Objek yang berbasis OBE, dengan mengadopsi prinsip perancangan sistem yang sistematis \cite{ref8}. Keterkaitan ini diwujudkan melalui:

\subsection{Alignment dengan CPL dan CPMK}

Setiap bab dalam buku ini dipetakan secara eksplisit ke Sub-CPMK yang berkontribusi pada pencapaian CPMK dan CPL. Struktur ini memastikan bahwa:
\begin{itemize}
  \item Materi pembelajaran fokus pada pencapaian kompetensi terukur
  \item Aktivitas pembelajaran mendukung pengembangan keterampilan yang diharapkan
  \item Asesmen mengukur pencapaian kompetensi secara objektif
\end{itemize}

\subsection{Integrasi Metode Pembelajaran}

Buku ini mengintegrasikan berbagai metode pembelajaran yang tercantum dalam RPS:
\begin{itemize}
  \item \textbf{Problem-Based Learning}: Studi kasus nyata dalam setiap bab
  \item \textbf{Project-Based Learning}: Proyek mini untuk integrasi konsep
  \item \textbf{Peer Review}: Aktivitas code review antar mahasiswa
  \item \textbf{Flipped Classroom}: Materi untuk dipelajari mandiri sebelum kelas
\end{itemize}

\subsection{Sistem Penilaian Terintegrasi}

Komponen asesmen dalam buku ini sejalan dengan sistem penilaian RPS:
\begin{itemize}
  \item Latihan untuk tugas individu (15\%)
  \item Kuis untuk evaluasi formatif (10\%)
  \item Aktivitas praktikum (15\%)
  \item Proyek kelompok (20\%)
  \item Persiapan UTS dan UAS (40\%)
\end{itemize}
