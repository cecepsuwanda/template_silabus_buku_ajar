\section{Konteks Kurikulum OBE}

\subsection{Apa itu Outcome-Based Education?}

\oop{Outcome-Based Education (OBE)} adalah pendekatan pembelajaran yang berfokus pada pencapaian hasil (\textit{outcomes}) yang terukur. Dalam OBE, proses pembelajaran dirancang secara sistematis untuk memastikan mahasiswa mencapai kompetensi yang telah ditetapkan.

\textbf{Prinsip Utama OBE:}
\begin{enumerate}
  \item \textbf{Clarity of Focus}: Fokus jelas pada apa yang harus dicapai mahasiswa
  \item \textbf{Designing Down}: Kurikulum dirancang mundur dari outcomes yang diinginkan
  \item \textbf{High Expectations}: Ekspektasi tinggi untuk semua mahasiswa
  \item \textbf{Expanded Opportunity}: Kesempatan beragam untuk mencapai outcomes
\end{enumerate}

\subsection{Implementasi OBE dalam Buku Ini}

Buku ini mengimplementasikan OBE melalui:

\begin{table}[h]
\centering
\begin{tabular}{|l|p{10cm}|}
\hline
\textbf{Komponen OBE} & \textbf{Implementasi dalam Buku} \\
\hline
Outcomes yang Jelas & Sub-CPMK eksplisit di setiap bab \\
\hline
Pembelajaran Terstruktur & Materi disusun dari dasar ke lanjut secara sistematis \\
\hline
Aktivitas Beragam & Latihan, studi kasus, proyek, code review \\
\hline
Asesmen Terukur & Rubrik penilaian yang jelas untuk setiap kompetensi \\
\hline
Feedback Berkelanjutan & Checklist untuk self-assessment \\
\hline
\end{tabular}
\caption{Implementasi OBE dalam Buku Ajar}
\end{table}

\subsection{Hierarki Capaian Pembelajaran}

\begin{konsep}
Dalam kurikulum OBE, capaian pembelajaran tersusun dalam hierarki:

\textbf{CPL (Capaian Pembelajaran Lulusan)} \\
$\downarrow$ \\
\textbf{CPMK (Capaian Pembelajaran Mata Kuliah)} \\
$\downarrow$ \\
\textbf{Sub-CPMK (Sub Capaian Pembelajaran Mata Kuliah)} \\
$\downarrow$ \\
\textbf{Indikator Pencapaian}

Setiap level berkontribusi pada level di atasnya, memastikan bahwa pembelajaran di tingkat mikro (per bab) mendukung pencapaian kompetensi di tingkat makro (lulusan).
\end{konsep}
