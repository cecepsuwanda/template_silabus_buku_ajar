\section{Tujuan Buku Ajar}

Buku ajar ini disusun dengan tujuan utama untuk membantu mahasiswa mencapai kompetensi dalam Pemrograman Berorientasi Objek melalui pendekatan \textbf{Outcome-Based Education (OBE)}. Tujuan spesifik buku ini adalah:

\begin{enumerate}
  \item Memberikan pemahaman mendalam tentang paradigma pemrograman berorientasi objek
  \item Mengembangkan kemampuan merancang solusi perangkat lunak menggunakan prinsip OOP
  \item Membangun keterampilan implementasi OOP dengan bahasa pemrograman Java
  \item Menumbuhkan kemampuan analisis dan evaluasi kualitas kode
  \item Memfasilitasi pencapaian CPL dan CPMK yang telah ditetapkan
\end{enumerate}

Setelah mempelajari buku ini secara menyeluruh, mahasiswa diharapkan mampu:
\begin{itemize}
  \item Menjelaskan konsep fundamental OOP (class, object, encapsulation, inheritance, polymorphism)
  \item Merancang sistem menggunakan UML diagrams
  \item Mengimplementasikan aplikasi Java dengan menerapkan prinsip SOLID
  \item Mengidentifikasi dan menerapkan design patterns yang tepat
  \item Menulis kode yang clean, maintainable, dan testable
\end{itemize}
