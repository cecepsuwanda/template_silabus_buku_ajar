\section{Asesmen Akhir Komprehensif}

\subsection{Petunjuk Umum}

Asesmen ini dirancang untuk mengukur pencapaian CPMK-1, CPMK-2, CPMK-3, dan CPMK-4 secara menyeluruh \cite{ref8}. Kerjakan dengan jujur dan mandiri.

\textbf{Alokasi Waktu}:
\begin{itemize}
  \item Bagian A (Pilihan Ganda): 30 menit
  \item Bagian B (Essay): 45 menit
  \item Bagian C (Analisis Kode): 45 menit
  \item Bagian D (Coding Challenge): 90 menit
\end{itemize}

\subsection{Bagian A: Pilihan Ganda (CPMK-1)}

\textbf{Petunjuk}: Pilih satu jawaban yang paling tepat.

\begin{enumerate}
  \item Manakah yang BUKAN merupakan pilar OOP?
  \begin{enumerate}
    \item Abstraksi
    \item Enkapsulasi
    \item Kompilasi
    \item Polimorfisme
  \end{enumerate}
  
  \item Dalam Java, keyword untuk mewarisi class adalah:
  \begin{enumerate}
    \item inherits
    \item extends
    \item implements
    \item super
  \end{enumerate}
  
  \item Method overriding terjadi ketika:
  \begin{enumerate}
    \item Dua method dalam class yang sama memiliki nama sama
    \item Subclass mendefinisikan ulang method dari superclass
    \item Method memiliki parameter yang berbeda
    \item Method dipanggil berkali-kali
  \end{enumerate}
  
  \item Interface dalam Java:
  \begin{enumerate}
    \item Dapat memiliki instance variables
    \item Semua methods harus abstract (sebelum Java 8)
    \item Dapat di-instantiate
    \item Hanya boleh memiliki satu method
  \end{enumerate}
  
  \item Prinsip SOLID yang menyatakan "class harus terbuka untuk extension tapi tertutup untuk modification" adalah:
  \begin{enumerate}
    \item Single Responsibility Principle
    \item Open/Closed Principle
    \item Liskov Substitution Principle
    \item Dependency Inversion Principle
  \end{enumerate}
  
  \item Design pattern yang memastikan hanya ada satu instance dari class adalah:
  \begin{enumerate}
    \item Factory Pattern
    \item Singleton Pattern
    \item Observer Pattern
    \item Strategy Pattern
  \end{enumerate}
  
  \item Dalam Java Collections, struktur data yang tidak mengizinkan duplikat adalah:
  \begin{enumerate}
    \item List
    \item Set
    \item Map
    \item Queue
  \end{enumerate}
  
  \item Exception yang harus di-handle dengan try-catch disebut:
  \begin{enumerate}
    \item RuntimeException
    \item Checked Exception
    \item Unchecked Exception
    \item Error
  \end{enumerate}
  
  \item Dalam UML Class Diagram, relasi "has-a" digambarkan dengan:
  \begin{enumerate}
    \item Generalization
    \item Association/Aggregation/Composition
    \item Dependency
    \item Realization
  \end{enumerate}

  \item Kelas yang digunakan untuk menulis objek ke file secara serialization adalah:
  \begin{enumerate}
    \item FileWriter
    \item ObjectOutputStream
    \item BufferedReader
    \item Scanner
  \end{enumerate}
  
  \item Annotation untuk menandai method sebagai test case dalam JUnit adalah:
  \begin{enumerate}
    \item @Test
    \item @TestCase
    \item @Unit
    \item @Assert
  \end{enumerate}

  \item Refactoring yang memecah method panjang menjadi beberapa method kecil disebut:
  \begin{enumerate}
    \item Extract Method
    \item Inline Method
    \item Replace Conditionals with Polymorphism
    \item Move Method
  \end{enumerate}
\end{enumerate}

\subsection{Bagian B: Essay (CPMK-1, CPMK-2)}

\textbf{Petunjuk}: Jawab dengan jelas dan lengkap.
\begin{enumerate}
  \item Jelaskan perbedaan antara abstract class dan interface dalam Java! Kapan sebaiknya menggunakan abstract class dan kapan menggunakan interface? Berikan contoh kasus untuk masing-masing.
  
  \item Jelaskan konsep polymorphism dalam OOP! Berikan contoh kode Java yang mendemonstrasikan compile-time polymorphism dan runtime polymorphism.
  
  \item Jelaskan 5 prinsip SOLID! Pilih salah satu prinsip dan berikan contoh kode yang melanggar prinsip tersebut, kemudian perbaiki kode tersebut.
  
  \item Gambarkan Class Diagram untuk sistem perpustakaan sederhana yang memiliki minimal 5 class dengan relasi yang tepat (association, aggregation, composition, inheritance).
  
  \item Jelaskan konsep Test-Driven Development (TDD)! Apa keuntungan menggunakan TDD dalam pengembangan software?

  \item Jelaskan contoh refactoring sederhana yang dapat Anda lakukan untuk memperbaiki code smell "Long Method".
\end{enumerate}

\subsection{Bagian C: Analisis Kode (CPMK-3, CPMK-4)}

\textbf{Petunjuk}: Analisis kode berikut dan jawab pertanyaan.

\begin{javacode}[caption={Kode untuk Dianalisis}]
public class BankAccount {
    public String accountNumber;
    public double balance;
    public String ownerName;
    
    public void deposit(double amount) {
        balance = balance + amount;
    }
    
    public void withdraw(double amount) {
        balance = balance - amount;
    }
    
    public double getBalance() {
        return balance;
    }
}

public class SavingsAccount extends BankAccount {
    public double interestRate;
    
    public void addInterest() {
        balance = balance + (balance * interestRate);
    }
}
\end{javacode}

\textbf{Pertanyaan}:
\begin{enumerate}
  \item Identifikasi minimal 5 masalah dalam kode di atas terkait dengan prinsip OOP dan best practices.
  \item Perbaiki kode tersebut dengan menerapkan encapsulation yang tepat.
  \item Tambahkan validasi yang diperlukan pada method \method{deposit} dan \method{withdraw}.
  \item Implementasikan constructor yang sesuai untuk kedua class.
  \item Apakah kode ini melanggar prinsip SOLID? Jelaskan!
\end{enumerate}

\subsection{Bagian D: Coding Challenge (CPMK-2, CPMK-3, 4)}

\textbf{Petunjuk}: Implementasikan sistem berikut dengan menerapkan semua konsep OOP yang telah dipelajari.

\textbf{Studi Kasus: Sistem Manajemen Toko Buku Online}

Buat sistem manajemen toko buku online dengan requirements berikut:

\textbf{Requirements}:
\begin{enumerate}
  \item \textbf{Class \class{Buku}}:
  \begin{itemize}
    \item Attributes: isbn, judul, penulis, penerbit, tahunTerbit, harga, stok
    \item Methods: getInfo(), updateStok(), hitungDiskon()
  \end{itemize}
  
  \item \textbf{Class \class{BukuFisik}} (extends \class{Buku}):
  \begin{itemize}
    \item Attribute tambahan: berat, dimensi
    \item Method: hitungOngkir()
  \end{itemize}
  
  \item \textbf{Class \class{Ebook}} (extends \class{Buku}):
  \begin{itemize}
    \item Attribute tambahan: ukuranFile, format
    \item Method: download()
  \end{itemize}
  
  \item \textbf{Class \class{Pelanggan}}:
  \begin{itemize}
    \item Attributes: idPelanggan, nama, email, alamat
    \item Methods: register(), updateProfile()
  \end{itemize}
  
  \item \textbf{Class \class{Keranjang}}:
  \begin{itemize}
    \item Attributes: daftarBuku (ArrayList), pelanggan
    \item Methods: tambahBuku(), hapusBuku(), hitungTotal(), checkout()
  \end{itemize}
  
  \item \textbf{Interface \class{Pembayaran}}:
  \begin{itemize}
    \item Methods: prosesPembayaran(), verifikasiPembayaran()
  \end{itemize}
  
  \item \textbf{Class \class{PembayaranKartuKredit}} dan \class{Pembayaran\-Transfer} (implements \class{Pembayaran})
\end{enumerate}

\textbf{Kriteria Penilaian}:
\begin{itemize}
  \item Penerapan encapsulation (private attributes, getter/setter)
  \item Penggunaan inheritance yang tepat
  \item Implementasi polymorphism
  \item Penggunaan interface
  \item Validasi data yang sesuai
  \item Exception handling
  \item Code quality (naming, comments, structure)
  \item Unit testing untuk minimal 3 methods
\end{itemize}
