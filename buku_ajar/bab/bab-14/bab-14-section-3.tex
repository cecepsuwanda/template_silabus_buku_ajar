\section{Tinjauan Pencapaian Kompetensi Secara Menyeluruh}

\subsection{Pemetaan Asesmen ke CPMK}

\begin{table}[h]
\centering
\begin{tabular}{|p{5cm}|p{8cm}|}
\hline
\textbf{CPMK} & \textbf{Diukur Melalui} \\
\hline
CPMK-1: Memahami konsep dasar OOP & Pilihan Ganda (1-4), Essay (1-2) \\
\hline
CPMK-2: Merancang solusi dengan UML & Essay (4), Coding Challenge (Design) \\
\hline
CPMK-3: Mengimplementasikan OOP dengan SOLID & Analisis Kode, Coding Challenge (Implementation) \\
\hline
CPMK-4: Menganalisis dan mengevaluasi kualitas kode & Analisis Kode, Coding Challenge (Quality) \\
\hline
\end{tabular}
\caption{Pemetaan Komponen Asesmen ke CPMK}
\end{table}

\subsection{Self-Assessment Checklist}

Sebelum mengerjakan asesmen akhir, pastikan Anda telah menguasai:

\begin{checklist}
  \item Konsep dasar OOP (class, object, encapsulation, inheritance, polymorphism)
  \item Perbedaan abstract class dan interface
  \item Cara membuat dan menggunakan UML diagrams
  \item Implementasi inheritance dan polymorphism dalam Java
  \item Exception handling yang tepat
  \item Prinsip-prinsip SOLID
  \item Design patterns dasar (Singleton, Factory, Observer)
  \item Java Collections Framework
  \item File I/O dan serialization
  \item Unit testing dengan JUnit
  \item Refactoring dan identifikasi code smell
  \item Best practices dalam penulisan kode Java
\end{checklist}
