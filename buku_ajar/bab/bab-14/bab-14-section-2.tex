\section{Rubrik Penilaian Komprehensif}

\subsection{Rubrik untuk Coding Challenge}

\begin{table}[h]
\centering
\small
\begin{tabular}{|>{\raggedright\arraybackslash}p{2.2cm}|>{\raggedright\arraybackslash}p{2.6cm}|>{\raggedright\arraybackslash}p{2.6cm}|>{\raggedright\arraybackslash}p{2.6cm}|>{\raggedright\arraybackslash}p{2.5cm}|}
\hline
\textbf{Kriteria} & \textbf{Excellent (4)} & \textbf{Good (3)} & \textbf{Fair (2)} & \textbf{Poor (1)} \\
\hline
Encapsulation & Semua attributes private dengan getter/setter yang tepat & Sebagian besar private & Beberapa public & Semua public \\
\hline
Inheritance & Hierarki class tepat, code reuse optimal & Inheritance digunakan dengan baik & Inheritance kurang optimal & Tidak menggunakan inheritance \\
\hline
Polymorphism & Men-de-mon-stra-si-kan compile-time dan runtime polymorphism & Menggunakan salah satu jenis polymorphism & Polymorphism minimal & Tidak ada polymorphism \\
\hline
Interface & Interface digunakan dengan tepat & Interface digunakan tapi kurang optimal & Interface ada tapi tidak efektif & Tidak menggunakan interface \\
\hline
Exception Handling & Comprehensive error handling & Error handling untuk kasus utama & Minimal error handling & Tidak ada error handling \\
\hline
Code Quality & Clean code, well-documented, follows conventions & Good structure, adequate comments & Basic structure, minimal comments & Poor structure, no comments \\
\hline
Unit Testing & Comprehensive tests, good coverage & Tests untuk fungsi utama & Minimal testing & No testing \\
\hline
\end{tabular}
\caption{Rubrik Penilaian Coding Challenge}
\end{table}

\subsection{Bobot Penilaian}

\begin{table}[h]
\centering
\begin{tabular}{|l|c|}
\hline
\textbf{Komponen} & \textbf{Bobot} \\
\hline
Bagian A: Pilihan Ganda & 20\% \\
\hline
Bagian B: Essay & 20\% \\
\hline
Bagian C: Analisis Kode & 20\% \\
\hline
Bagian D: Coding Challenge & 40\% \\
\hline
\textbf{Total} & \textbf{100\%} \\
\hline
\end{tabular}
\caption{Matriks Bobot Penilaian Akhir}
\end{table}
