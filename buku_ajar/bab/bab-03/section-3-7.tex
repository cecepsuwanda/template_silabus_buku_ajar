\section{Static vs Instance Members}

\subsection{Instance Members}

\textbf{Instance members} (attributes dan methods) adalah milik object. Setiap object memiliki copy sendiri.

\subsection{Static Members}

\textbf{Static members} adalah milik class, bukan object. Dibagi oleh semua object.

\begin{javacode}[caption={Static vs Instance}]
public class Mahasiswa {
    // Instance variable (setiap object punya copy sendiri)
    private String nama;
    private double ipk;
    
    // Static variable (dibagi semua object)
    private static int jumlahMahasiswa = 0;
    private static String namaUniversitas = "Universitas Lorem Ipsum";
    
    public Mahasiswa(String nama, double ipk) {
        this.nama = nama;
        this.ipk = ipk;
        jumlahMahasiswa++;  // Increment setiap object dibuat
    }
    
    // Instance method
    public void tampilkanInfo() {
        System.out.println("Nama: " + nama);
        System.out.println("IPK: " + ipk);
    }
    
    // Static method
    public static int getJumlahMahasiswa() {
        return jumlahMahasiswa;
    }
    
    public static String getNamaUniversitas() {
        return namaUniversitas;
    }
}

// Penggunaan
public class Main {
    public static void main(String[] args) {
        // Akses static member tanpa object
        System.out.println("Universitas: " + Mahasiswa.getNamaUniversitas());
        
        Mahasiswa mhs1 = new Mahasiswa("Budi", 3.5);
        Mahasiswa mhs2 = new Mahasiswa("Ani", 3.8);
        
        // Akses static member melalui class name
        System.out.println("Total mahasiswa: " + Mahasiswa.getJumlahMahasiswa());
        // Output: Total mahasiswa: 2
    }
}
\end{javacode}

\begin{catatan}
\textbf{Best Practice}:
\begin{itemize}
  \item Akses static members melalui class name, bukan object
  \item Static methods tidak bisa mengakses instance variables secara langsung
  \item Gunakan static untuk utility methods atau constants
\end{itemize}
\end{catatan}
