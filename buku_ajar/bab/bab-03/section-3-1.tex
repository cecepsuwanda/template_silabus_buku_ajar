\section{Definisi Class dan Object}

\subsection{Apa itu Class?}

\begin{konsep}
\textbf{Class} adalah blueprint atau template untuk membuat objek. Class mendefinisikan attributes (data) dan methods (behavior) yang akan dimiliki oleh objek.

Analogi: Class seperti cetak biru rumah, sedangkan object adalah rumah yang dibangun berdasarkan cetak biru tersebut.
\end{konsep}

Dalam Java, class didefinisikan dengan keyword \keyword{class}:

\begin{javacode}[caption={Struktur Dasar Class}]
public class NamaClass {
    // Attributes (instance variables)
    tipeData namaAttribute;
    
    // Constructor
    public NamaClass() {
        // Inisialisasi
    }
    
    // Methods
    public void namaMethod() {
        // Implementasi
    }
}
\end{javacode}

\subsection{Apa itu Object?}

\begin{konsep}
\textbf{Object} adalah instance (perwujudan) dari class. Object memiliki state (nilai attributes) dan behavior (methods) yang didefinisikan oleh class-nya.
\end{konsep}

Membuat object dalam Java menggunakan keyword \keyword{new}:

\begin{javacode}[caption={Membuat Object}]
NamaClass namaObject = new NamaClass();
\end{javacode}

\subsection{Hubungan Class dan Object}

\begin{table}[h]
\centering
\begin{tabular}{|p{3cm}|p{5cm}|p{5cm}|}
\hline
\textbf{Aspek} & \textbf{Class} & \textbf{Object} \\
\hline
Definisi & Template/Blueprint & Instance dari class \\
\hline
Keberadaan & Logical entity & Physical entity \\
\hline
Memory & Tidak menggunakan memory & Menggunakan memory \\
\hline
Jumlah & Satu class & Banyak object dari satu class \\
\hline
Keyword & \code{class} & \code{new} \\
\hline
\end{tabular}
\end{table}
