\section{Constructor}

\subsection{Apa itu Constructor?}

\begin{konsep}
\textbf{Constructor} adalah method khusus yang dipanggil saat object dibuat. Constructor digunakan untuk menginisialisasi attributes object.

Ciri-ciri constructor:
\begin{itemize}
  \item Nama sama dengan nama class
  \item Tidak memiliki return type (bahkan bukan \code{void})
  \item Dipanggil otomatis saat object dibuat dengan \keyword{new}
\end{itemize}
\end{konsep}

\begin{javacode}[caption={Contoh Constructor}]
public class Mahasiswa {
    String nim;
    String nama;
    double ipk;
    
    // Constructor tanpa parameter (default constructor)
    public Mahasiswa() {
        nim = "0000";
        nama = "Unknown";
        ipk = 0.0;
    }
    
    // Constructor dengan parameter
    public Mahasiswa(String nim, String nama) {
        this.nim = nim;
        this.nama = nama;
        this.ipk = 0.0;
    }
    
    // Constructor dengan semua parameter
    public Mahasiswa(String nim, String nama, double ipk) {
        this.nim = nim;
        this.nama = nama;
        this.ipk = ipk;
    }
}
\end{javacode}

\subsection{Constructor Overloading}

Class dapat memiliki multiple constructors dengan parameter yang berbeda. Ini disebut \oop{constructor overloading}.
