\section{Object Creation dan Instantiation}

\subsection{Membuat Object}

Proses membuat object disebut \oop{instantiation}. Dalam Java, gunakan keyword \keyword{new}:

\begin{javacode}[caption={Membuat dan Menggunakan Object}]
public class Main {
    public static void main(String[] args) {
        // Membuat object dengan default constructor
        Mahasiswa mhs1 = new Mahasiswa();
        
        // Membuat object dengan parameterized constructor
        Mahasiswa mhs2 = new Mahasiswa("123456", "Budi Santoso");
        
        // Mengakses attributes (jika public)
        mhs2.ipk = 3.75;
        
        // Memanggil methods
        mhs2.tampilkanInfo();
        System.out.println("Predikat: " + mhs2.getPredikat());
        
        // Cek kelulusan
        if (mhs2.lulus()) {
            System.out.println(mhs2.nama + " dinyatakan LULUS");
        }
    }
}
\end{javacode}

\subsection{Memory Allocation}

Ketika object dibuat:
\begin{enumerate}
  \item Memory dialokasikan di \textbf{heap}
  \item Reference variable disimpan di \textbf{stack}
  \item Constructor dipanggil untuk inisialisasi
\end{enumerate}

\begin{catatan}
Jika tidak ada constructor yang didefinisikan, Java otomatis menyediakan \textbf{default constructor} tanpa parameter yang menginisialisasi attributes dengan nilai default (0, null, false).
\end{catatan}
