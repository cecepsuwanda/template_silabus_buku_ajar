\section{Methods (Behavior)}

\subsection{Definisi Methods}

\textbf{Methods} adalah fungsi yang didefinisikan di dalam class untuk menentukan behavior dari object.

\begin{javacode}[caption={Contoh Methods}]
public class Mahasiswa {
    String nim;
    String nama;
    double ipk;
    
    // Method untuk menampilkan info
    public void tampilkanInfo() {
        System.out.println("NIM: " + nim);
        System.out.println("Nama: " + nama);
        System.out.println("IPK: " + ipk);
    }
    
    // Method untuk mengecek kelulusan
    public boolean lulus() {
        return ipk >= 2.0;
    }
    
    // Method untuk mendapatkan predikat
    public String getPredikat() {
        if (ipk >= 3.5) return "Cum Laude";
        else if (ipk >= 3.0) return "Sangat Memuaskan";
        else if (ipk >= 2.5) return "Memuaskan";
        else return "Cukup";
    }
}
\end{javacode}

\subsection{Jenis Methods}

\begin{enumerate}
  \item \textbf{Accessor Methods (Getter)}: Mengambil nilai attribute
  \item \textbf{Mutator Methods (Setter)}: Mengubah nilai attribute
  \item \textbf{Utility Methods}: Melakukan operasi tertentu
\end{enumerate}
