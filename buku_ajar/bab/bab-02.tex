\documentclass[c:/Matakuliah/template_silabus_buku_ajar/buku_ajar/main.tex]{subfiles}
\ifSubfilesClassLoaded{\setcounter{chapter}{1}}{}
\begin{document}

\chapter{Landasan Teori dan Konsep Dasar OOP}

\begin{subcpmk}
  \item Sub-CPMK 1.1: Menjelaskan perbedaan antara pemrograman prosedural dan berorientasi objek serta mengidentifikasi karakteristik utama paradigma OOP
\end{subcpmk}

% ============================================================
% MATERI POKOK
% ============================================================
\section{Pendahuluan dan Konsep Dasar}
\begin{subcpmk}
  \item Sub-CPMK 2.1: ...
\end{subcpmk}
Materi section 2.1. Isi bab kedua, section pertama. Gantikan dengan uraian sesuai mata kuliah dan capaian pembelajaran.
\begin{latihan}
  \item ...
  \item ...
\end{latihan}

\section{Tinjauan dan Contoh}
\begin{subcpmk}
  \item Sub-CPMK 2.2: ...
\end{subcpmk}
Materi section 2.2. Tinjauan lanjutan dan contoh penerapan. Sesuaikan dengan CPMK bab ini.
\begin{latihan}
  \item ...
  \item ...
\end{latihan}

\section{Konsep Dasar OOP}

OOP dibangun di atas empat pilar utama:

\subsection{1. Abstraksi (Abstraction)}

\begin{konsep}
\textbf{Abstraksi} adalah proses menyembunyikan detail implementasi dan hanya menampilkan fungsionalitas kepada pengguna.

Contoh: Ketika menggunakan mobil, Anda hanya perlu tahu cara menggunakan setir, pedal gas, dan rem. Anda tidak perlu tahu detail mesin internal.
\end{konsep}

Dalam OOP, abstraksi dicapai melalui:
\begin{itemize}
  \item Abstract classes
  \item Interfaces
  \item Encapsulation
\end{itemize}

\subsection{2. Enkapsulasi (Encapsulation)}

\begin{konsep}
\textbf{Enkapsulasi} adalah pembungkusan data dan method yang beroperasi pada data tersebut dalam satu unit (class), serta menyembunyikan detail internal dari luar.
\end{konsep}

Manfaat enkapsulasi:
\begin{itemize}
  \item \textbf{Data Hiding}: Melindungi data dari akses tidak sah
  \item \textbf{Modularity}: Kode lebih terorganisir
  \item \textbf{Flexibility}: Mudah mengubah implementasi internal
  \item \textbf{Maintainability}: Lebih mudah dipelihara
\end{itemize}

\subsection{3. Pewarisan (Inheritance)}

\begin{konsep}
\textbf{Inheritance} adalah mekanisme di mana class baru (subclass) dapat mewarisi properties dan methods dari class yang sudah ada (superclass).
\end{konsep}

Manfaat inheritance:
\begin{itemize}
  \item \textbf{Code Reusability}: Tidak perlu menulis ulang kode yang sama
  \item \textbf{Hierarchical Classification}: Organisasi class yang terstruktur
  \item \textbf{Extensibility}: Mudah menambah fitur baru
\end{itemize}

\subsection{4. Polimorfisme (Polymorphism)}

\begin{konsep}
\textbf{Polimorfisme} adalah kemampuan objek untuk mengambil banyak bentuk. Dalam OOP, ini berarti satu interface dapat digunakan untuk tipe data yang berbeda.
\end{konsep}

Jenis polimorfisme:
\begin{itemize}
  \item \textbf{Compile-time Polymorphism}: Method overloading
  \item \textbf{Runtime Polymorphism}: Method overriding
\end{itemize}

\section{Keuntungan dan Tantangan OOP}

\subsection{Keuntungan OOP}

\begin{enumerate}
  \item \textbf{Modularitas}: Kode terorganisir dalam class-class yang independen
  \item \textbf{Reusability}: Code reuse melalui inheritance dan composition
  \item \textbf{Maintainability}: Lebih mudah menemukan dan memperbaiki bug
  \item \textbf{Scalability}: Mudah menambah fitur baru tanpa mengubah kode existing
  \item \textbf{Real-world Modeling}: Lebih natural dalam memodelkan dunia nyata
  \item \textbf{Data Security}: Enkapsulasi melindungi data
  \item \textbf{Collaboration}: Tim dapat bekerja pada class yang berbeda secara paralel
\end{enumerate}

\subsection{Tantangan OOP}

\begin{enumerate}
  \item \textbf{Learning Curve}: Membutuhkan pemahaman konsep yang lebih dalam
  \item \textbf{Overhead}: Bisa lebih lambat untuk program sederhana
  \item \textbf{Complexity}: Bisa menjadi terlalu kompleks jika tidak dirancang dengan baik
  \item \textbf{Design Effort}: Membutuhkan perencanaan dan desain yang matang
\end{enumerate}

\begin{catatan}
OOP bukan solusi untuk semua masalah. Untuk program sederhana atau script kecil, pendekatan prosedural mungkin lebih efisien. Gunakan OOP ketika:
\begin{itemize}
  \item Program cukup besar dan kompleks
  \item Membutuhkan code reuse yang tinggi
  \item Banyak developer bekerja pada proyek yang sama
  \item Sistem perlu mudah di-maintain dan di-extend
\end{itemize}
\end{catatan}

\section{Peta Konsep OOP}

Berikut adalah peta konsep yang menunjukkan hubungan antar konsep OOP:

\begin{center}
\textbf{PEMROGRAMAN BERORIENTASI OBJEK}\\
$\downarrow$\\
\textbf{4 Pilar Utama}\\
$\downarrow$\\
\begin{tabular}{cccc}
Abstraksi & Enkapsulasi & Inheritance & Polimorfisme\\
\end{tabular}\\
$\downarrow$\\
\textbf{Diimplementasikan melalui}\\
$\downarrow$\\
\begin{tabular}{ccc}
Class & Object & Interface\\
\end{tabular}\\
$\downarrow$\\
\textbf{Menghasilkan}\\
$\downarrow$\\
\begin{tabular}{ccc}
Modular & Reusable & Maintainable\\
\end{tabular}\\
Code
\end{center}


% ============================================================
% AKTIVITAS PEMBELAJARAN
% ============================================================

\begin{aktivitas}
  \item \textbf{Diskusi Kelompok}: Identifikasi 5 contoh objek di dunia nyata (misalnya: mobil, HP, ATM) dan tentukan attributes dan behaviors yang bisa dimodelkan dalam OOP.
  
  \item \textbf{Analisis Kode}: Diberikan kode prosedural untuk sistem perpustakaan sederhana, identifikasi bagian mana yang bisa diubah menjadi class dalam OOP.
  
  \item \textbf{Perbandingan}: Tulis program sederhana (misalnya kalkulator) dalam dua versi: prosedural dan OOP. Bandingkan kelebihan dan kekurangan masing-masing.
  
  \item \textbf{Mind Mapping}: Buat mind map yang menghubungkan 4 pilar OOP dengan contoh implementasi konkret dalam Java.
\end{aktivitas}

% ============================================================
% LATIHAN DAN REFLEKSI
% ============================================================

\begin{latihan}
  \item Jelaskan perbedaan utama antara pemrograman prosedural dan OOP! Berikan contoh kasus di mana OOP lebih cocok digunakan.
  
  \item Sebutkan dan jelaskan 4 pilar OOP! Berikan contoh sederhana untuk masing-masing pilar.
  
  \item Mengapa enkapsulasi penting dalam OOP? Apa yang terjadi jika semua data dalam class bersifat public?
  
  \item Berikan 3 keuntungan dan 2 tantangan dalam menggunakan OOP.
  
  \item Identifikasi objek-objek dalam sistem e-commerce (misalnya Tokopedia). Sebutkan minimal 5 class yang mungkin ada beserta attributes dan methods-nya.
  
  \item Apakah semua program harus ditulis dengan OOP? Jelaskan kapan sebaiknya menggunakan OOP dan kapan tidak.
  
  \item \textbf{Refleksi}: Bagaimana pemahaman Anda tentang OOP sebelum dan sesudah mempelajari bab ini? Konsep mana yang paling sulit dipahami?
\end{latihan}

% ============================================================
% ASESMEN
% ============================================================

\begin{asesmen}
\textbf{Instrumen Penilaian untuk Sub-CPMK 1.1}

\textbf{A. Pilihan Ganda}

\begin{enumerate}
  \item Manakah yang BUKAN merupakan pilar OOP?
  \begin{enumerate}
    \item Abstraksi
    \item Enkapsulasi
    \item Kompilasi
    \item Polimorfisme
  \end{enumerate}
  
  \item Dalam OOP, data dan method yang beroperasi pada data tersebut digabung dalam satu unit yang disebut:
  \begin{enumerate}
    \item Function
    \item Module
    \item Class
    \item Library
  \end{enumerate}
  
  \item Keuntungan utama enkapsulasi adalah:
  \begin{enumerate}
    \item Membuat program lebih cepat
    \item Melindungi data dari akses tidak sah
    \item Mengurangi ukuran file
    \item Mempermudah kompilasi
  \end{enumerate}
\end{enumerate}

\textbf{B. Essay}

\begin{enumerate}
  \item Jelaskan dengan kata-kata Anda sendiri apa yang dimaksud dengan "objek" dalam OOP dan berikan 2 contoh objek dari dunia nyata beserta attributes dan behaviors-nya.
  
  \item Bandingkan pendekatan prosedural dan OOP dalam menyelesaikan masalah pengelolaan data mahasiswa. Mana yang lebih cocok dan mengapa?
\end{enumerate}

\textbf{Rubrik Penilaian}: Lihat Lampiran A
\end{asesmen}

% ============================================================
% CHECKLIST KOMPETENSI
% ============================================================

\begin{checklist}
  \item Saya dapat menjelaskan perbedaan antara pemrograman prosedural dan OOP
  \item Saya dapat menyebutkan dan menjelaskan 4 pilar OOP
  \item Saya dapat memberikan contoh konkret untuk setiap pilar OOP
  \item Saya dapat mengidentifikasi kapan sebaiknya menggunakan OOP
  \item Saya memahami keuntungan dan tantangan dalam menggunakan OOP
  \item Saya dapat mengidentifikasi objek, attributes, dan methods dari dunia nyata
\end{checklist}

% ============================================================
% RANGKUMAN
% ============================================================

\begin{rangkuman}
Bab ini membahas landasan teori Pemrograman Berorientasi Objek, termasuk evolusi paradigma pemrograman, perbandingan antara pendekatan prosedural dan OOP, serta 4 pilar utama OOP.

\textbf{Poin Kunci:}
\begin{itemize}
  \item OOP adalah paradigma yang berfokus pada objek yang mengenkapsulasi data dan perilaku
  \item 4 Pilar OOP: Abstraksi, Enkapsulasi, Inheritance, Polimorfisme
  \item OOP menawarkan modularitas, reusability, dan maintainability yang lebih baik
  \item OOP lebih cocok untuk sistem besar dan kompleks
  \item Pemahaman konsep dasar OOP adalah fondasi untuk mempelajari implementasi OOP dalam Java
\end{itemize}

\textbf{Kata Kunci}: \oop{OOP}, \oop{Class}, \oop{Object}, \oop{Abstraksi}, \oop{Enkapsulasi}, \oop{Inheritance}, \oop{Polimorfisme}, \oop{Modularitas}
\end{rangkuman}

\ifSubfilesClassLoaded{
  \renewcommand{\bibname}{Daftar Pustaka}
  \bibliographystyle{plain}
  \bibliography{c:/Matakuliah/template_silabus_buku_ajar/buku_ajar/references}
}{}
\end{document}
