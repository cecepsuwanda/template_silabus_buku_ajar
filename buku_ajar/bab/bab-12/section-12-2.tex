\section{File I/O dan Serialization}

\subsection{Membaca dan Menulis File}
File I/O memungkinkan aplikasi menyimpan data secara permanen.

\begin{javacode}[caption={Contoh Menulis File}]
import java.io.FileWriter;
import java.io.IOException;

public class TulisFile {
    public static void main(String[] args) throws IOException {
        try (FileWriter writer = new FileWriter("catatan.txt")) {
            writer.write("Belajar File I/O");
        }
    }
}
\end{javacode}

\subsection{Serialization}
Serialization mengubah objek menjadi byte stream agar dapat disimpan atau dikirim.

% ============================================================
% AKTIVITAS PEMBELAJARAN
% ============================================================

\begin{aktivitas}
  \item \textbf{Collections}: Buat program manajemen daftar tugas dengan ArrayList.
  \item \textbf{Map}: Simpan pasangan NIM dan nama mahasiswa menggunakan HashMap.
  \item \textbf{File I/O}: Simpan data mahasiswa ke file teks dan baca kembali.
  \item \textbf{Serialization}: Simpan objek \class{Mahasiswa} ke file menggunakan \code{ObjectOutputStream}.
\end{aktivitas}

% ============================================================
% LATIHAN DAN REFLEKSI
% ============================================================

\begin{latihan}
  \item Jelaskan perbedaan List, Set, dan Map.
  \item Buat contoh penggunaan generics pada class \class{Kotak<T>}.
  \item Buat program yang membaca file CSV sederhana.
  \item Jelaskan manfaat serialization dan kapan sebaiknya digunakan.
  \item \textbf{Refleksi}: Bagaimana Collections membantu Anda menulis kode yang lebih rapi?
\end{latihan}

% ============================================================
% ASESMEN
% ============================================================

\begin{asesmen}
\textbf{Instrumen Penilaian untuk Sub-CPMK 3.1}

\textbf{A. Pilihan Ganda}
\begin{enumerate}
  \item Struktur data yang tidak mengizinkan duplikat adalah:
  \begin{enumerate}
    \item List
    \item Set
    \item Map
    \item Queue
  \end{enumerate}
  \item Try-with-resources berguna untuk:
  \begin{enumerate}
    \item Menghapus file
    \item Menutup resource otomatis
    \item Mengubah format file
    \item Mempercepat loop
  \end{enumerate}
\end{enumerate}

\textbf{B. Tugas Praktik}
\begin{itemize}
  \item Buat aplikasi pencatatan transaksi sederhana menggunakan List dan file teks.
\end{itemize}

\textbf{Rubrik Penilaian}: Lihat Lampiran A
\end{asesmen}

% ============================================================
% CHECKLIST KOMPETENSI
% ============================================================

\begin{checklist}
  \item Saya dapat membedakan List, Set, dan Map
  \item Saya dapat menggunakan generics untuk keamanan tipe
  \item Saya dapat membaca dan menulis file sederhana
  \item Saya memahami konsep serialization
  \item Saya dapat memilih koleksi yang sesuai dengan kebutuhan
\end{checklist}

% ============================================================
% RANGKUMAN
% ============================================================

\begin{rangkuman}
Collections memudahkan pengelolaan data dinamis, generics meningkatkan keamanan tipe, dan file I/O memungkinkan data disimpan secara permanen sesuai standar Java \cite{ref7}. Serialization membantu penyimpanan objek secara langsung.

\textbf{Kata Kunci}: \oop{Collections}, \oop{Generics}, \oop{List}, \oop{Map}, \oop{File I/O}, \oop{Serialization}
\end{rangkuman}
