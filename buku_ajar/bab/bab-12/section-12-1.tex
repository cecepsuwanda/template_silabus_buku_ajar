\section{Java Collections Framework dan Generics}
\begin{subcpmk}
  \item Sub-CPMK 3.1: Mengimplementasikan struktur data dengan Collections dan Generics
\end{subcpmk}

\textbf{Java Collections Framework} adalah arsitektur terpadu untuk merepresentasikan dan memanipulasi koleksi object \cite{ref4}.

\subsection{Jenis Collections}
\begin{itemize}
  \item \textbf{List}: urutan terjaga, boleh duplikat (ArrayList, LinkedList)
  \item \textbf{Set}: tidak boleh duplikat (HashSet, TreeSet)
  \item \textbf{Map}: pasangan key-value (HashMap, TreeMap)
  \item \textbf{Queue}: antrian (PriorityQueue)
\end{itemize}

\subsection{Generics}
Generics membuat koleksi lebih aman terhadap tipe data dan mengurangi casting.

\begin{javacode}[caption={Contoh Collections dan Generics}]
import java.util.ArrayList;
import java.util.List;

public class DemoList {
    public static void main(String[] args) {
        List<String> nama = new ArrayList<>();
        nama.add("Ayu");
        nama.add("Bima");
        for (String n : nama) {
            System.out.println(n);
        }
    }
}
\end{javacode}
