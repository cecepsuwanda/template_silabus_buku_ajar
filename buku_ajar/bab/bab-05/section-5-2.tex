\section{Overriding, \texttt{super}, dan Desain Hierarki}

\subsection{Method Overriding}
Subclass dapat mendefinisikan ulang method dari superclass. Gunakan anotasi \code{@Override} untuk memperjelas niat.

\begin{javacode}[caption={Contoh Method Overriding}]
class Kendaraan {
    public void nyalakan() {
        System.out.println("Kendaraan dinyalakan");
    }
}

class Mobil extends Kendaraan {
    @Override
    public void nyalakan() {
        System.out.println("Mobil dinyalakan dengan kunci");
    }
}
\end{javacode}

\subsection{Penggunaan \texttt{super}}
\code{super} digunakan untuk memanggil constructor atau method dari superclass.

\begin{javacode}[caption={Penggunaan super}]
class Dosen {
    protected String nama;

    public Dosen(String nama) {
        this.nama = nama;
    }
}

class DosenTetap extends Dosen {
    private String nidn;

    public DosenTetap(String nama, String nidn) {
        super(nama);
        this.nidn = nidn;
    }
}
\end{javacode}

\subsection{Composition vs Inheritance}
Gunakan inheritance jika relasi \textit{is-a} kuat. Jika tidak, gunakan composition (\textit{has-a}) untuk fleksibilitas.

\begin{catatan}
Contoh: \class{Mobil} \textit{has-a} \class{Mesin}. Lebih tepat menggunakan composition.
\end{catatan}

% ============================================================
% AKTIVITAS PEMBELAJARAN
% ============================================================

\begin{aktivitas}
  \item \textbf{Hierarki Kelas}: Buat hierarki class untuk sistem akademik (Person $\rightarrow$ Mahasiswa/Dosen).
  \item \textbf{Overriding}: Implementasikan method \method{hitungGaji} berbeda di class \class{Pegawai} dan \class{Manajer}.
  \item \textbf{Diskusi}: Identifikasi contoh \textit{is-a} dan \textit{has-a} dalam aplikasi e-commerce.
  \item \textbf{Code Review}: Cari potensi penyalahgunaan inheritance pada kode yang diberikan dosen.
\end{aktivitas}

% ============================================================
% LATIHAN DAN REFLEKSI
% ============================================================

\begin{latihan}
  \item Jelaskan perbedaan inheritance dan composition.
  \item Buat class \class{Hewan} dan subclass \class{Kucing}. Tambahkan method \method{suara} dan override di subclass.
  \item Kapan penggunaan inheritance dapat menyebabkan masalah desain? Berikan contoh.
  \item Buat class \class{Pegawai} dan subclass \class{PegawaiKontrak}. Tambahkan atribut dan method spesifik.
  \item \textbf{Refleksi}: Apa tantangan terbesar Anda saat menyusun hierarki kelas?
\end{latihan}

% ============================================================
% ASESMEN
% ============================================================

\begin{asesmen}
\textbf{Instrumen Penilaian untuk Sub-CPMK 3.1}

\textbf{A. Pilihan Ganda}
\begin{enumerate}
  \item Keyword untuk melakukan inheritance di Java adalah:
  \begin{enumerate}
    \item inherit
    \item extends
    \item implements
    \item override
  \end{enumerate}
  \item Method overriding berarti:
  \begin{enumerate}
    \item Dua method dengan nama sama dalam class yang sama
    \item Subclass mendefinisikan ulang method superclass
    \item Method dipanggil berulang
    \item Method private diakses dari luar class
  \end{enumerate}
\end{enumerate}

\textbf{B. Tugas Praktik}
\begin{itemize}
  \item Buat hierarki class \class{Produk} $\rightarrow$ \class{ProdukDigital} dan \class{ProdukFisik} dengan atribut dan method yang sesuai.
  \item Demonstrasikan overriding pada method \method{hitungHargaAkhir}.
\end{itemize}

\textbf{Rubrik Penilaian}: Lihat Lampiran A
\end{asesmen}

% ============================================================
% CHECKLIST KOMPETENSI
% ============================================================

\begin{checklist}
  \item Saya memahami konsep inheritance dan relasi \textit{is-a}
  \item Saya dapat membuat subclass dengan \keyword{extends}
  \item Saya mampu menerapkan method overriding
  \item Saya memahami penggunaan \code{super}
  \item Saya dapat memilih inheritance atau composition dengan tepat
\end{checklist}

% ============================================================
% RANGKUMAN
% ============================================================

\begin{rangkuman}
Inheritance memungkinkan class baru mewarisi perilaku dan data dari class lain, mendukung code reuse dan desain hierarki \cite{ref4}. Overriding dan \code{super} adalah kunci untuk menyesuaikan perilaku subclass.

\textbf{Kata Kunci}: \oop{Inheritance}, \oop{extends}, \oop{overriding}, \oop{super}, \oop{is-a}, \oop{composition}
\end{rangkuman}
