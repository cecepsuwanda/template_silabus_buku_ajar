\section{Konsep Inheritance dan Relasi "is-a"}
\begin{subcpmk}
  \item Sub-CPMK 3.1: Mengimplementasikan inheritance dan method overriding
\end{subcpmk}

\subsection{Konsep Pewarisan}
\begin{konsep}
Inheritance memungkinkan kita untuk membuat class baru berdasarkan class yang sudah ada, mewarisi semua attributes dan methods-nya \cite{ref4}. Relasi ini dikenal sebagai relasi \textit{is-a}.
\end{konsep}

Keuntungan utama inheritance:
\begin{itemize}
  \item \textbf{Code reuse}: mengurangi duplikasi kode
  \item \textbf{Extensibility}: menambah fitur dengan subclass
  \item \textbf{Polymorphism}: memungkinkan penggunaan reference superclass
\end{itemize}

\subsection{Keyword \texttt{extends}}
Dalam Java, inheritance dideklarasikan dengan keyword \keyword{extends}.

\begin{javacode}[caption={Inheritance Sederhana}]
class Kendaraan {
    protected String merk;

    public Kendaraan(String merk) {
        this.merk = merk;
    }

    public void nyalakan() {
        System.out.println("Kendaraan dinyalakan");
    }
}

class Mobil extends Kendaraan {
    private int jumlahPintu;

    public Mobil(String merk, int jumlahPintu) {
        super(merk);
        this.jumlahPintu = jumlahPintu;
    }
}
\end{javacode}

\subsection{Hierarki Kelas}
Hierarki kelas membantu kita mengorganisasi konsep dalam struktur bertingkat:

\begin{figure}[h]
\centering
\begin{tikzpicture}[node distance=1.5cm]
  \node (kendaraan) [class] {Kendaraan};
  \node (mobil) [class, below left=of kendaraan] {Mobil};
  \node (motor) [class, below right=of kendaraan] {Motor};
  
  \draw [arrow] (mobil) -- (kendaraan);
  \draw [arrow] (motor) -- (kendaraan);
\end{tikzpicture}
\caption{Contoh Hierarki Kelas (Relasi is-a)}
\end{figure}
