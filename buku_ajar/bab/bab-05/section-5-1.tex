\section{Konsep Inheritance dan Relasi "is-a"}
\begin{subcpmk}
  \item Sub-CPMK 3.1: Mengimplementasikan inheritance dan method overriding
\end{subcpmk}

\begin{konsep}
\textbf{Inheritance} adalah mekanisme pewarisan yang memungkinkan class baru (subclass) mewarisi attribute dan method dari class yang sudah ada (superclass). Relasi ini dikenal sebagai relasi \textit{is-a}.
\end{konsep}

Keuntungan utama inheritance:
\begin{itemize}
  \item \textbf{Code reuse}: mengurangi duplikasi kode
  \item \textbf{Extensibility}: menambah fitur dengan subclass
  \item \textbf{Polymorphism}: memungkinkan penggunaan reference superclass
\end{itemize}

\subsection{Keyword \texttt{extends}}
Dalam Java, inheritance dideklarasikan dengan keyword \keyword{extends}.

\begin{javacode}[caption={Inheritance Sederhana}]
class Kendaraan {
    protected String merk;

    public Kendaraan(String merk) {
        this.merk = merk;
    }

    public void nyalakan() {
        System.out.println("Kendaraan dinyalakan");
    }
}

class Mobil extends Kendaraan {
    private int jumlahPintu;

    public Mobil(String merk, int jumlahPintu) {
        super(merk);
        this.jumlahPintu = jumlahPintu;
    }
}
\end{javacode}

\subsection{Hierarki Kelas}
Hierarki kelas membantu kita mengorganisasi konsep dalam struktur bertingkat, misalnya:
\begin{itemize}
  \item Kendaraan $\rightarrow$ Mobil, Motor
  \item Pegawai $\rightarrow$ Dosen, Staf
\end{itemize}
