\section{Interface dan Perbandingannya dengan Abstract Class}

\subsection{Konsep Interface}
Interface mendefinisikan kontrak yang harus diimplementasikan oleh class.

\begin{javacode}[caption={Contoh Interface}]
interface Pembayaran {
    void proses();
    void verifikasi();
}

class PembayaranKartu implements Pembayaran {
    @Override
    public void proses() {
        System.out.println("Proses kartu kredit");
    }

    @Override
    public void verifikasi() {
        System.out.println("Verifikasi kartu kredit");
    }
}
\end{javacode}

\subsection{Perbandingan Abstract Class vs Interface}
\begin{table}[h]
\centering
\begin{tabular}{|p{3cm}|p{5cm}|p{5.5cm}|}
\hline
\textbf{Aspek} & \textbf{Abstract Class} & \textbf{Interface} \\
\hline
Instansiasi & Tidak dapat diinstansiasi & Tidak dapat diinstansiasi \\
\hline
Inheritance & Single inheritance & Multiple inheritance of type \\
\hline
State & Dapat memiliki attribute & Hanya constant (public static final) \\
\hline
Method & Abstract dan konkret & Abstract, default, static \\
\hline
\end{tabular}
\caption{Perbandingan Abstract Class dan Interface}
\end{table}

% ============================================================
% AKTIVITAS PEMBELAJARAN
% ============================================================

\begin{aktivitas}
\sloppy
  \item \textbf{Diskusi}: Tentukan kapan menggunakan abstract class dan kapan interface.
  \item \textbf{Implementasi}: Buat interface \class{Notifikasi}. Lalu buat class \class{EmailNotifikasi} dan \class{SMSNotifikasi}.
  \item \textbf{Studi Kasus}: Rancang abstract class \class{Bentuk} dan implementasi \class{Lingkaran}, \class{Segitiga}.
  \item \textbf{Review}: Analisis kode yang menggunakan inheritance berlebihan dan usulkan perbaikan dengan interface.
\end{aktivitas}

% ============================================================
% LATIHAN DAN REFLEKSI
% ============================================================

\begin{latihan}
  \item Jelaskan perbedaan utama antara abstract class dan interface.
  \item Buat abstract class \class{Transportasi} dan subclass \class{Kereta}, \class{Bus}.
  \item Buat interface \class{Diskon} dan implementasikan pada class \class{Produk}.
  \item Jelaskan manfaat default method pada interface.
  \item \textbf{Refleksi}: Bagaimana Anda menentukan pilihan desain antara abstract class dan interface?
\end{latihan}

% ============================================================
% ASESMEN
% ============================================================

\begin{asesmen}
\textbf{Instrumen Penilaian untuk Sub-CPMK 3.2}

\textbf{A. Pilihan Ganda}
\begin{enumerate}
  \item Interface di Java dapat diimplementasikan oleh:
  \begin{enumerate}
    \item Satu class saja
    \item Banyak class
    \item Hanya abstract class
    \item Hanya class final
  \end{enumerate}
  \item Abstract class dapat memiliki:
  \begin{enumerate}
    \item Hanya method abstract
    \item Hanya method static
    \item Method abstract dan method konkret
    \item Hanya constructor private
  \end{enumerate}
\end{enumerate}

\textbf{B. Tugas Praktik}
\begin{itemize}
  \item Rancang interface \class{Layanan} dan dua implementasi berbeda. Tunjukkan penggunaan polymorphism.
\end{itemize}

\textbf{Rubrik Penilaian}: Lihat Lampiran A
\end{asesmen}

% ============================================================
% CHECKLIST KOMPETENSI
% ============================================================

\begin{checklist}
  \item Saya memahami konsep abstract class
  \item Saya memahami konsep interface
  \item Saya dapat memilih antara abstract class dan interface
  \item Saya dapat mengimplementasikan interface di Java
  \item Saya dapat menggunakan polymorphism dengan abstract class dan interface
\end{checklist}

% ============================================================
% RANGKUMAN
% ============================================================

\begin{rangkuman}
Abstract class menyediakan dasar perilaku dan state bersama, sedangkan interface menyediakan kontrak untuk perilaku \cite{ref4}. Keduanya mendukung polymorphism dan desain yang fleksibel.

\textbf{Kata Kunci}: \oop{Abstract Class}, \oop{Interface}, \oop{implements}, \oop{default method}, \oop{polymorphism}
\end{rangkuman}
