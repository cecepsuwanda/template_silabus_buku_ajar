\section{Abstract Class dan Abstraksi}
\begin{subcpmk}
  \item Sub-CPMK 3.2: Menerapkan polymorphism dalam desain aplikasi
\end{subcpmk}

\begin{konsep}
\textbf{Abstraksi} melalui abstract class dan interface adalah mekanisme utama untuk mencapai desain yang fleksibel dalam Java \cite{ref4}. Abstract class tidak dapat diinstansiasi dan dapat memiliki method abstrak serta konkret.
\end{konsep}

\subsection{Karakteristik Abstract Class}
\begin{itemize}
  \item Dideklarasikan dengan keyword \keyword{abstract}
  \item Dapat memiliki atribut dan constructor
  \item Dapat memiliki method abstract dan non-abstract
\end{itemize}

\begin{javacode}[caption={Contoh Abstract Class}]
abstract class Pegawai {
    protected String nama;

    public Pegawai(String nama) {
        this.nama = nama;
    }

    public abstract double hitungGaji();

    public void tampilkanInfo() {
        System.out.println("Nama: " + nama);
    }
}

class PegawaiTetap extends Pegawai {
    private double gajiBulanan;

    public PegawaiTetap(String nama, double gajiBulanan) {
        super(nama);
        this.gajiBulanan = gajiBulanan;
    }

    @Override
    public double hitungGaji() {
        return gajiBulanan;
    }
}
\end{javacode}

\begin{figure}[h]
\centering
\begin{tikzpicture}[node distance=2cm]
  \node (pegawai) [class, fill=gray!20] {\textit{Pegawai}};
  \node (tetap) [class, below=of pegawai] {PegawaiTetap};
  
  \draw [arrow] (tetap) -- (pegawai);
\end{tikzpicture}
\caption{Hierarki Class Pegawai (Abstract)}
\end{figure}

\subsection{Kapan Menggunakan Abstract Class}
Gunakan abstract class ketika:
\begin{itemize}
  \item Ada perilaku umum yang ingin diwariskan
  \item Ada method yang wajib diimplementasikan subclass
  \item Dibutuhkan constructor atau state bersama
\end{itemize}
