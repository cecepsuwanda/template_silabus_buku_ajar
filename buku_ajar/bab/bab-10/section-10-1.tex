\section{SRP, OCP, dan LSP}
\begin{subcpmk}
  \item Sub-CPMK 4.1: Menganalisis code smell dan melakukan refactoring
\end{subcpmk}

\begin{konsep}
\textbf{SOLID} adalah lima prinsip desain yang membuat perangkat lunak lebih mudah dipahami, fleksibel, dan maintainable \cite{ref3}.
\end{konsep}

\subsection{Single Responsibility Principle (SRP)}
Satu class sebaiknya memiliki satu alasan untuk berubah.

\subsection{Open/Closed Principle (OCP)}
Class harus terbuka untuk extension tetapi tertutup untuk modification.

\subsection{Liskov Substitution Principle (LSP)}
Objek subclass harus dapat menggantikan objek superclass tanpa merusak perilaku program.

\begin{catatan}
Jika class sering berubah karena banyak alasan, itu adalah tanda pelanggaran SRP.
\end{catatan}
