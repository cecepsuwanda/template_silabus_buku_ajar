\section{Use Case dan Class Diagram}
\begin{subcpmk}
  \item Sub-CPMK 2.1: Membuat use case diagram untuk sistem sederhana
  \item Sub-CPMK 2.2: Merancang class diagram dengan relasi yang tepat
\end{subcpmk}

\subsection{Unified Modeling Language (UML)}
Unified Modeling Language (UML) adalah bahasa standar untuk memvisualisasikan, menspesifikasikan, membangun, dan mendokumentasikan sistem perangkat lunak \cite{ref8}.

\subsection{Use Case Diagram}
Use case diagram menggambarkan interaksi aktor dengan sistem pada level fungsional.

\textbf{Komponen utama}:
\begin{itemize}
  \item \textbf{Actor}: pihak yang berinteraksi dengan sistem
  \item \textbf{Use case}: layanan yang disediakan sistem
  \item \textbf{Boundary system}: batasan sistem
\end{itemize}

\subsection{Class Diagram}
Class diagram menggambarkan struktur statis sistem: class, attribute, method, dan relasi.

\textbf{Relasi utama}:
\begin{itemize}
  \item \textbf{Association}: relasi umum antar class
  \item \textbf{Aggregation}: relasi has-a yang lemah
  \item \textbf{Composition}: relasi has-a yang kuat
  \item \textbf{Inheritance}: relasi is-a
\end{itemize}

\begin{catatan}
Class diagram yang baik membantu tim memahami struktur sistem sebelum implementasi kode.
\end{catatan}
