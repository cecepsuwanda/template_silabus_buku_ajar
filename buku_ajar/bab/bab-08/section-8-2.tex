\section{Sequence Diagram dan Studi Kasus}

\subsection{Sequence Diagram}
Sequence diagram menggambarkan alur komunikasi antar objek dalam urutan waktu.

\textbf{Elemen penting}:
\begin{itemize}
  \item \textbf{Lifeline}: objek yang terlibat
  \item \textbf{Message}: pesan yang dikirim antar objek
  \item \textbf{Activation}: periode eksekusi suatu objek
\end{itemize}

\subsection{Contoh Studi Kasus}
Studi kasus peminjaman buku perpustakaan:
\begin{enumerate}
  \item Mahasiswa memilih buku
  \item Sistem mengecek ketersediaan
  \item Sistem mencatat transaksi
  \item Sistem memperbarui status buku
\end{enumerate}

% ============================================================
% AKTIVITAS PEMBELAJARAN
% ============================================================

\begin{aktivitas}
  \item \textbf{Use Case}: Buat use case diagram untuk sistem pemesanan makanan online.
  \item \textbf{Class Diagram}: Rancang class diagram untuk sistem perpustakaan dengan minimal 5 class.
  \item \textbf{Sequence Diagram}: Buat sequence diagram untuk proses login aplikasi.
  \item \textbf{Review}: Bandingkan class diagram Anda dengan rekan dan diskusikan perbaikannya.
\end{aktivitas}

% ============================================================
% LATIHAN DAN REFLEKSI
% ============================================================

\begin{latihan}
  \item Jelaskan perbedaan use case diagram dan class diagram.
  \item Sebutkan minimal 3 relasi pada class diagram dan jelaskan masing-masing.
  \item Buat use case diagram sederhana untuk sistem parkir kampus.
  \item Buat sequence diagram untuk proses checkout e-commerce.
  \item \textbf{Refleksi}: Bagian mana dari UML yang paling menantang bagi Anda?
\end{latihan}

% ============================================================
% ASESMEN
% ============================================================

\begin{asesmen}
\textbf{Instrumen Penilaian untuk Sub-CPMK 2.1 dan 2.2}

\textbf{A. Pilihan Ganda}
\begin{enumerate}
  \item Relasi \textit{composition} pada class diagram menunjukkan:
  \begin{enumerate}
    \item Relasi is-a
    \item Relasi has-a yang kuat
    \item Relasi dependensi sementara
    \item Relasi tanpa kepemilikan
  \end{enumerate}
  \item Sequence diagram berfokus pada:
  \begin{enumerate}
    \item Struktur statis sistem
    \item Urutan interaksi antar objek
    \item Penjelasan kebutuhan pengguna
    \item Deskripsi database
  \end{enumerate}
\end{enumerate}

\textbf{B. Tugas Praktik}
\begin{itemize}
  \item Rancang use case diagram dan class diagram untuk sistem peminjaman alat laboratorium.
  \item Tambahkan sequence diagram untuk proses peminjaman.
\end{itemize}

\textbf{Rubrik Penilaian}: Lihat Lampiran A
\end{asesmen}

% ============================================================
% CHECKLIST KOMPETENSI
% ============================================================

\begin{checklist}
  \item Saya dapat membuat use case diagram untuk sistem sederhana
  \item Saya dapat merancang class diagram dengan relasi yang tepat
  \item Saya memahami fungsi sequence diagram
  \item Saya dapat menerjemahkan kebutuhan ke UML
  \item Saya mampu membaca dan meninjau diagram UML
\end{checklist}

% ============================================================
% RANGKUMAN
% ============================================================

\begin{rangkuman}
UML membantu memodelkan sistem secara visual sesuai dengan standar industri \cite{ref8}. Use case diagram memetakan kebutuhan, class diagram menggambarkan struktur, dan sequence diagram menunjukkan alur interaksi.

\textbf{Kata Kunci}: \oop{UML}, \oop{Use Case}, \oop{Class Diagram}, \oop{Sequence Diagram}, \oop{Association}
\end{rangkuman}
