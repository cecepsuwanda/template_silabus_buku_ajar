\section{Custom Exception dan Best Practice}

\subsection{Membuat Custom Exception}
Custom exception digunakan untuk mewakili kesalahan yang spesifik pada domain aplikasi.

\begin{javacode}[caption={Custom Exception}]
class SaldoTidakCukupException extends Exception {
    public SaldoTidakCukupException(String pesan) {
        super(pesan);
    }
}

class Rekening {
    private double saldo;

    public void tarik(double jumlah) throws SaldoTidakCukupException {
        if (jumlah > saldo) {
            throw new SaldoTidakCukupException("Saldo tidak mencukupi");
        }
        saldo -= jumlah;
    }
}
\end{javacode}

\subsection{Finally dan Try-with-Resources}
Gunakan \keyword{finally} untuk memastikan resource ditutup. Di Java modern, gunakan try-with-resources.

% ============================================================
% AKTIVITAS PEMBELAJARAN
% ============================================================

\begin{aktivitas}
  \item \textbf{Eksperimen}: Buat program yang memicu \code{NumberFormatException} dan tangani dengan try-catch.
  \item \textbf{Custom Exception}: Buat exception \class{StokHabisException} pada sistem inventori.
  \item \textbf{Diskusi}: Kapan sebaiknya melempar exception dan kapan mengembalikan nilai default?
  \item \textbf{Refactoring}: Ganti penanganan error manual dengan try-with-resources pada contoh file I/O.
\end{aktivitas}

% ============================================================
% LATIHAN DAN REFLEKSI
% ============================================================

\begin{latihan}
  \item Jelaskan perbedaan checked dan unchecked exception.
  \item Buat program input angka yang aman dari kesalahan format input.
  \item Buat custom exception untuk validasi umur minimal 17 tahun.
  \item Jelaskan kapan \keyword{finally} tetap dieksekusi.
  \item \textbf{Refleksi}: Bagaimana exception handling membantu kualitas aplikasi Anda?
\end{latihan}

% ============================================================
% ASESMEN
% ============================================================

\begin{asesmen}
\textbf{Instrumen Penilaian untuk Sub-CPMK 3.1}

\textbf{A. Pilihan Ganda}
\begin{enumerate}
  \item Exception yang wajib ditangani adalah:
  \begin{enumerate}
    \item Checked exception
    \item Runtime exception
    \item Error
    \item Logical error
  \end{enumerate}
  \item Try-with-resources digunakan untuk:
  \begin{enumerate}
    \item Menghindari overloading
    \item Menutup resource secara otomatis
    \item Menghapus object
    \item Mempercepat program
  \end{enumerate}
\end{enumerate}

\textbf{B. Tugas Praktik}
\begin{itemize}
  \item Buat class \class{Login} yang melempar exception jika password salah tiga kali.
\end{itemize}

\textbf{Rubrik Penilaian}: Lihat Lampiran A
\end{asesmen}

% ============================================================
% CHECKLIST KOMPETENSI
% ============================================================

\begin{checklist}
  \item Saya memahami konsep exception dan error
  \item Saya dapat menggunakan try-catch dengan benar
  \item Saya mampu membuat custom exception
  \item Saya memahami penggunaan finally dan try-with-resources
  \item Saya dapat menentukan kapan melempar exception
\end{checklist}

% ============================================================
% RANGKUMAN
% ============================================================

\begin{rangkuman}
Exception handling menjaga program tetap stabil saat terjadi kesalahan sesuai spesifikasi Java \cite{ref7}. Dengan custom exception dan try-with-resources, aplikasi menjadi lebih robust dan mudah dipelihara.

\textbf{Kata Kunci}: \oop{Exception}, \oop{try-catch}, \oop{checked}, \oop{unchecked}, \oop{try-with-resources}
\end{rangkuman}
