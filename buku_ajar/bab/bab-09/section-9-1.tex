\section{Konsep Exception dan Try-Catch}
\begin{subcpmk}
  \item Sub-CPMK 3.1: Mengimplementasikan exception handling dalam program Java
\end{subcpmk}

\begin{konsep}
\textbf{Exception handling} adalah mekanisme untuk menangani kondisi tidak terduga selama runtime, memastikan program tetap berjalan stabil \cite{ref4}. Exception handling membantu program tetap stabil dan memberikan pesan kesalahan yang jelas.
\end{konsep}

\subsection{Jenis Exception}
\begin{itemize}
  \item \textbf{Checked Exception}: wajib ditangani (contoh: \code{IOException})
  \item \textbf{Unchecked Exception}: turunan \code{RuntimeException}
  \item \textbf{Error}: kesalahan serius yang biasanya tidak ditangani
\end{itemize}

\subsection{Struktur Try-Catch}

\begin{javacode}[caption={Try-Catch Sederhana}]
public class DemoException {
    public static void main(String[] args) {
        try {
            int hasil = 10 / 0;
            System.out.println(hasil);
        } catch (ArithmeticException e) {
            System.out.println("Terjadi kesalahan: " + e.getMessage());
        }
    }
}
\end{javacode}
