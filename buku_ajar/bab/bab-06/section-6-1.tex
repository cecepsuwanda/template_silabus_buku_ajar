\section{Polymorphism dan Method Overloading}
\begin{subcpmk}
  \item Sub-CPMK 3.2: Menerapkan polymorphism dalam desain aplikasi
\end{subcpmk}

\begin{konsep}
\textbf{Polymorphism} adalah kemampuan objek untuk tampil dalam banyak bentuk. Dalam Java, polymorphism muncul dalam dua bentuk: compile-time (overloading) dan runtime (overriding).
\end{konsep}

\subsection{Compile-time Polymorphism (Overloading)}
Method overloading terjadi ketika terdapat beberapa method dengan nama sama, tetapi parameter berbeda.

\begin{javacode}[caption={Contoh Overloading}]
class Kalkulator {
    public int tambah(int a, int b) {
        return a + b;
    }

    public double tambah(double a, double b) {
        return a + b;
    }

    public int tambah(int a, int b, int c) {
        return a + b + c;
    }
}
\end{javacode}

\subsection{Keuntungan Overloading}
\begin{itemize}
  \item Memudahkan penggunaan API karena nama method konsisten
  \item Mengurangi kompleksitas nama method
  \item Membuat kode lebih mudah dibaca
\end{itemize}
