\section{Runtime Polymorphism dan Dynamic Dispatch}

\subsection{Overriding dan Dynamic Binding}
Runtime polymorphism terjadi saat method yang dipanggil ditentukan pada saat runtime berdasarkan objek aktual.

\begin{javacode}[caption={Runtime Polymorphism}]
class Bentuk {
    public double luas() {
        return 0.0;
    }
}

class Lingkaran extends Bentuk {
    private double r;

    public Lingkaran(double r) {
        this.r = r;
    }

    @Override
    public double luas() {
        return Math.PI * r * r;
    }
}

class Persegi extends Bentuk {
    private double s;

    public Persegi(double s) {
        this.s = s;
    }

    @Override
    public double luas() {
        return s * s;
    }
}

public class Demo {
    public static void main(String[] args) {
        Bentuk b1 = new Lingkaran(7);
        Bentuk b2 = new Persegi(4);
        System.out.println(b1.luas());
        System.out.println(b2.luas());
    }
}
\end{javacode}

\subsection{Polymorphism pada Interface}
Polymorphism juga terjadi ketika menggunakan reference bertipe interface.

% ============================================================
% AKTIVITAS PEMBELAJARAN
% ============================================================

\begin{aktivitas}
  \item \textbf{Eksperimen Overloading}: Buat class \class{Printer} dengan method \method{cetak} untuk tipe data berbeda.
  \item \textbf{Dynamic Dispatch}: Buat array bertipe superclass yang berisi beberapa subclass dan panggil method yang dioverride.
  \item \textbf{Studi Kasus}: Implementasikan interface \class{Pembayaran} dengan beberapa jenis pembayaran.
  \item \textbf{Diskusi}: Jelaskan manfaat polymorphism dalam pengembangan framework.
\end{aktivitas}

% ============================================================
% LATIHAN DAN REFLEKSI
% ============================================================

\begin{latihan}
  \item Apa perbedaan overloading dan overriding?
  \item Buat class \class{Hewan} dengan method \method{suara}. Buat subclass \class{Anjing} dan \class{Kucing} yang mengoverride method tersebut.
  \item Buat contoh penggunaan interface yang menunjukkan polymorphism.
  \item Mengapa polymorphism penting untuk extensibility aplikasi?
  \item \textbf{Refleksi}: Dalam proyek Anda, bagian mana yang paling terbantu oleh polymorphism?
\end{latihan}

% ============================================================
% ASESMEN
% ============================================================

\begin{asesmen}
\textbf{Instrumen Penilaian untuk Sub-CPMK 3.2}

\textbf{A. Pilihan Ganda}
\begin{enumerate}
  \item Runtime polymorphism terjadi karena:
  \begin{enumerate}
    \item Overloading
    \item Overriding
    \item Constructor
    \item Static method
  \end{enumerate}
  \item Method yang dipanggil pada runtime ditentukan oleh:
  \begin{enumerate}
    \item Tipe reference
    \item Tipe objek aktual
    \item Nama method
    \item Jumlah parameter
  \end{enumerate}
\end{enumerate}

\textbf{B. Tugas Praktik}
\begin{itemize}
  \item Buat hierarki class \class{Notifikasi} dengan subclass \class{Email} dan \class{SMS}. Demonstrasikan polymorphism saat mengirim notifikasi.
\end{itemize}

\textbf{Rubrik Penilaian}: Lihat Lampiran A
\end{asesmen}

% ============================================================
% CHECKLIST KOMPETENSI
% ============================================================

\begin{checklist}
  \item Saya memahami konsep polymorphism
  \item Saya dapat membuat method overloading
  \item Saya dapat menerapkan method overriding
  \item Saya memahami dynamic dispatch pada runtime
  \item Saya dapat menggunakan polymorphism dengan interface
\end{checklist}

% ============================================================
% RANGKUMAN
% ============================================================

\begin{rangkuman}
\sloppy
Polymorphism memungkinkan satu interface digunakan untuk banyak bentuk objek \cite{ref4}. Overloading terjadi di compile-time, sedangkan overriding terjadi di runtime melalui dynamic dispatch.

\textbf{Kata Kunci}: \oop{Polymorphism}, \oop{overloading}, \oop{overriding}, \oop{dynamic dispatch}, \oop{interface}
\end{rangkuman}
