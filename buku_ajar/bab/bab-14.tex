\documentclass[../main.tex]{subfiles}
\begin{document}

\chapter{Evaluasi, Refleksi, dan Integrasi Kompetensi}

Bab ini berisi asesmen komprehensif untuk mengukur pencapaian seluruh CPMK yang telah dipelajari sepanjang mata kuliah Pemrograman Berorientasi Objek.

\section{Asesmen Akhir Komprehensif}

\subsection{Petunjuk Umum}

Asesmen ini dirancang untuk mengukur pencapaian CPMK-1, CPMK-2, CPMK-3, dan CPMK-4 secara menyeluruh \cite{ref8}. Kerjakan dengan jujur dan mandiri.

\textbf{Alokasi Waktu}:
\begin{itemize}
  \item Bagian A (Pilihan Ganda): 30 menit
  \item Bagian B (Essay): 45 menit
  \item Bagian C (Analisis Kode): 45 menit
  \item Bagian D (Coding Challenge): 90 menit
\end{itemize}

\subsection{Bagian A: Pilihan Ganda (CPMK-1)}

\textbf{Petunjuk}: Pilih satu jawaban yang paling tepat.

\begin{enumerate}
  \item Manakah yang BUKAN merupakan pilar OOP?
  \begin{enumerate}
    \item Abstraksi
    \item Enkapsulasi
    \item Kompilasi
    \item Polimorfisme
  \end{enumerate}
  
  \item Dalam Java, keyword untuk mewarisi class adalah:
  \begin{enumerate}
    \item inherits
    \item extends
    \item implements
    \item super
  \end{enumerate}
  
  \item Method overriding terjadi ketika:
  \begin{enumerate}
    \item Dua method dalam class yang sama memiliki nama sama
    \item Subclass mendefinisikan ulang method dari superclass
    \item Method memiliki parameter yang berbeda
    \item Method dipanggil berkali-kali
  \end{enumerate}
  
  \item Interface dalam Java:
  \begin{enumerate}
    \item Dapat memiliki instance variables
    \item Semua methods harus abstract (sebelum Java 8)
    \item Dapat di-instantiate
    \item Hanya boleh memiliki satu method
  \end{enumerate}
  
  \item Prinsip SOLID yang menyatakan "class harus terbuka untuk extension tapi tertutup untuk modification" adalah:
  \begin{enumerate}
    \item Single Responsibility Principle
    \item Open/Closed Principle
    \item Liskov Substitution Principle
    \item Dependency Inversion Principle
  \end{enumerate}
  
  \item Design pattern yang memastikan hanya ada satu instance dari class adalah:
  \begin{enumerate}
    \item Factory Pattern
    \item Singleton Pattern
    \item Observer Pattern
    \item Strategy Pattern
  \end{enumerate}
  
  \item Dalam Java Collections, struktur data yang tidak mengizinkan duplikat adalah:
  \begin{enumerate}
    \item List
    \item Set
    \item Map
    \item Queue
  \end{enumerate}
  
  \item Exception yang harus di-handle dengan try-catch disebut:
  \begin{enumerate}
    \item RuntimeException
    \item Checked Exception
    \item Unchecked Exception
    \item Error
  \end{enumerate}
  
  \item Dalam UML Class Diagram, relasi "has-a" digambarkan dengan:
  \begin{enumerate}
    \item Generalization
    \item Association/Aggregation/Composition
    \item Dependency
    \item Realization
  \end{enumerate}

  \item Kelas yang digunakan untuk menulis objek ke file secara serialization adalah:
  \begin{enumerate}
    \item FileWriter
    \item ObjectOutputStream
    \item BufferedReader
    \item Scanner
  \end{enumerate}
  
  \item Annotation untuk menandai method sebagai test case dalam JUnit adalah:
  \begin{enumerate}
    \item @Test
    \item @TestCase
    \item @Unit
    \item @Assert
  \end{enumerate}

  \item Refactoring yang memecah method panjang menjadi beberapa method kecil disebut:
  \begin{enumerate}
    \item Extract Method
    \item Inline Method
    \item Replace Conditionals with Polymorphism
    \item Move Method
  \end{enumerate}
\end{enumerate}

\subsection{Bagian B: Essay (CPMK-1, CPMK-2)}

\textbf{Petunjuk}: Jawab dengan jelas dan lengkap.
\begin{enumerate}
  \item Jelaskan perbedaan antara abstract class dan interface dalam Java! Kapan sebaiknya menggunakan abstract class dan kapan menggunakan interface? Berikan contoh kasus untuk masing-masing.
  
  \item Jelaskan konsep polymorphism dalam OOP! Berikan contoh kode Java yang mendemonstrasikan compile-time polymorphism dan runtime polymorphism.
  
  \item Jelaskan 5 prinsip SOLID! Pilih salah satu prinsip dan berikan contoh kode yang melanggar prinsip tersebut, kemudian perbaiki kode tersebut.
  
  \item Gambarkan Class Diagram untuk sistem perpustakaan sederhana yang memiliki minimal 5 class dengan relasi yang tepat (association, aggregation, composition, inheritance).
  
  \item Jelaskan konsep Test-Driven Development (TDD)! Apa keuntungan menggunakan TDD dalam pengembangan software?

  \item Jelaskan contoh refactoring sederhana yang dapat Anda lakukan untuk memperbaiki code smell "Long Method".
\end{enumerate}

\subsection{Bagian C: Analisis Kode (CPMK-3, CPMK-4)}

\textbf{Petunjuk}: Analisis kode berikut dan jawab pertanyaan.

\begin{javacode}[caption={Kode untuk Dianalisis}]
public class BankAccount {
    public String accountNumber;
    public double balance;
    public String ownerName;
    
    public void deposit(double amount) {
        balance = balance + amount;
    }
    
    public void withdraw(double amount) {
        balance = balance - amount;
    }
    
    public double getBalance() {
        return balance;
    }
}

public class SavingsAccount extends BankAccount {
    public double interestRate;
    
    public void addInterest() {
        balance = balance + (balance * interestRate);
    }
}
\end{javacode}

\textbf{Pertanyaan}:
\begin{enumerate}
  \item Identifikasi minimal 5 masalah dalam kode di atas terkait dengan prinsip OOP dan best practices.
  \item Perbaiki kode tersebut dengan menerapkan encapsulation yang tepat.
  \item Tambahkan validasi yang diperlukan pada method \method{deposit} dan \method{withdraw}.
  \item Implementasikan constructor yang sesuai untuk kedua class.
  \item Apakah kode ini melanggar prinsip SOLID? Jelaskan!
\end{enumerate}

\subsection{Bagian D: Coding Challenge (CPMK-2, CPMK-3, 4)}

\textbf{Petunjuk}: Implementasikan sistem berikut dengan menerapkan semua konsep OOP yang telah dipelajari.

\textbf{Studi Kasus: Sistem Manajemen Toko Buku Online}

Buat sistem manajemen toko buku online dengan requirements berikut:

\textbf{Requirements}:
\begin{enumerate}
  \item \textbf{Class \class{Buku}}:
  \begin{itemize}
    \item Attributes: isbn, judul, penulis, penerbit, tahunTerbit, harga, stok
    \item Methods: getInfo(), updateStok(), hitungDiskon()
  \end{itemize}
  
  \item \textbf{Class \class{BukuFisik}} (extends \class{Buku}):
  \begin{itemize}
    \item Attribute tambahan: berat, dimensi
    \item Method: hitungOngkir()
  \end{itemize}
  
  \item \textbf{Class \class{Ebook}} (extends \class{Buku}):
  \begin{itemize}
    \item Attribute tambahan: ukuranFile, format
    \item Method: download()
  \end{itemize}
  
  \item \textbf{Class \class{Pelanggan}}:
  \begin{itemize}
    \item Attributes: idPelanggan, nama, email, alamat
    \item Methods: register(), updateProfile()
  \end{itemize}
  
  \item \textbf{Class \class{Keranjang}}:
  \begin{itemize}
    \item Attributes: daftarBuku (ArrayList), pelanggan
    \item Methods: tambahBuku(), hapusBuku(), hitungTotal(), checkout()
  \end{itemize}
  
  \item \textbf{Interface \class{Pembayaran}}:
  \begin{itemize}
    \item Methods: prosesPembayaran(), verifikasiPembayaran()
  \end{itemize}
  
  \item \textbf{Class \class{PembayaranKartuKredit}} dan \class{Pembayaran\-Transfer} (implements \class{Pembayaran})
\end{enumerate}

\textbf{Kriteria Penilaian}:
\begin{itemize}
  \item Penerapan encapsulation (private attributes, getter/setter)
  \item Penggunaan inheritance yang tepat
  \item Implementasi polymorphism
  \item Penggunaan interface
  \item Validasi data yang sesuai
  \item Exception handling
  \item Code quality (naming, comments, structure)
  \item Unit testing untuk minimal 3 methods
\end{itemize}

\section{Rubrik Penilaian Komprehensif}

\subsection{Rubrik untuk Coding Challenge}

\begin{table}[h]
\centering
\small
\begin{tabular}{|>{\raggedright\arraybackslash}p{2.2cm}|>{\raggedright\arraybackslash}p{2.6cm}|>{\raggedright\arraybackslash}p{2.6cm}|>{\raggedright\arraybackslash}p{2.6cm}|>{\raggedright\arraybackslash}p{2.5cm}|}
\hline
\textbf{Kriteria} & \textbf{Excellent (4)} & \textbf{Good (3)} & \textbf{Fair (2)} & \textbf{Poor (1)} \\
\hline
Encapsulation & Semua attributes private dengan getter/setter yang tepat & Sebagian besar private & Beberapa public & Semua public \\
\hline
Inheritance & Hierarki class tepat, code reuse optimal & Inheritance digunakan dengan baik & Inheritance kurang optimal & Tidak menggunakan inheritance \\
\hline
Polymorphism & Men-de-mon-stra-si-kan compile-time dan runtime polymorphism & Menggunakan salah satu jenis polymorphism & Polymorphism minimal & Tidak ada polymorphism \\
\hline
Interface & Interface digunakan dengan tepat & Interface digunakan tapi kurang optimal & Interface ada tapi tidak efektif & Tidak menggunakan interface \\
\hline
Exception Handling & Comprehensive error handling & Error handling untuk kasus utama & Minimal error handling & Tidak ada error handling \\
\hline
Code Quality & Clean code, well-documented, follows conventions & Good structure, adequate comments & Basic structure, minimal comments & Poor structure, no comments \\
\hline
Unit Testing & Comprehensive tests, good coverage & Tests untuk fungsi utama & Minimal testing & No testing \\
\hline
\end{tabular}
\caption{Rubrik Penilaian Coding Challenge}
\end{table}

\subsection{Bobot Penilaian}

\begin{table}[h]
\centering
\begin{tabular}{|l|c|}
\hline
\textbf{Komponen} & \textbf{Bobot} \\
\hline
Bagian A: Pilihan Ganda & 20\% \\
\hline
Bagian B: Essay & 20\% \\
\hline
Bagian C: Analisis Kode & 20\% \\
\hline
Bagian D: Coding Challenge & 40\% \\
\hline
\textbf{Total} & \textbf{100\%} \\
\hline
\end{tabular}
\caption{Matriks Bobot Penilaian Akhir}
\end{table}

\section{Tinjauan Pencapaian Kompetensi Secara Menyeluruh}

\subsection{Pemetaan Asesmen ke CPMK}

\begin{table}[h]
\centering
\begin{tabular}{|p{5cm}|p{8cm}|}
\hline
\textbf{CPMK} & \textbf{Diukur Melalui} \\
\hline
CPMK-1: Memahami konsep dasar OOP & Pilihan Ganda (1-4), Essay (1-2) \\
\hline
CPMK-2: Merancang solusi dengan UML & Essay (4), Coding Challenge (Design) \\
\hline
CPMK-3: Mengimplementasikan OOP dengan SOLID & Analisis Kode, Coding Challenge (Implementation) \\
\hline
CPMK-4: Menganalisis dan mengevaluasi kualitas kode & Analisis Kode, Coding Challenge (Quality) \\
\hline
\end{tabular}
\caption{Pemetaan Komponen Asesmen ke CPMK}
\end{table}

\subsection{Self-Assessment Checklist}

Sebelum mengerjakan asesmen akhir, pastikan Anda telah menguasai:

\begin{checklist}
  \item Konsep dasar OOP (class, object, encapsulation, inheritance, polymorphism)
  \item Perbedaan abstract class dan interface
  \item Cara membuat dan menggunakan UML diagrams
  \item Implementasi inheritance dan polymorphism dalam Java
  \item Exception handling yang tepat
  \item Prinsip-prinsip SOLID
  \item Design patterns dasar (Singleton, Factory, Observer)
  \item Java Collections Framework
  \item File I/O dan serialization
  \item Unit testing dengan JUnit
  \item Refactoring dan identifikasi code smell
  \item Best practices dalam penulisan kode Java
\end{checklist}

\section{Rekomendasi Tindak Lanjut Pembelajaran}

\subsection{Untuk Mahasiswa yang Sudah Menguasai Semua CPMK}

Jika Anda telah menguasai semua materi dengan baik, pertimbangkan untuk:

\begin{enumerate}
  \item \textbf{Proyek Portofolio}: Buat aplikasi Java lengkap yang mendemonstrasikan semua konsep OOP
  \item \textbf{Eksplorasi Framework}: Pelajari Spring Framework atau Jakarta EE
  \item \textbf{Design Patterns Lanjut}: Pelajari 23 GoF Design Patterns secara mendalam
  \item \textbf{Clean Code}: Baca buku "Clean Code" oleh Robert C. Martin
  \item \textbf{Kontribusi Open Source}: Berkontribusi pada proyek Java open source
  \item \textbf{Sertifikasi}: Persiapkan Oracle Certified Professional Java Programmer
\end{enumerate}

\subsection{Untuk Mahasiswa yang Masih Perlu Perbaikan}

Jika ada CPMK yang belum dikuasai dengan baik:

\begin{enumerate}
  \item \textbf{Review Materi}: Baca ulang bab terkait dengan lebih teliti
  \item \textbf{Praktik Lebih Banyak}: Kerjakan lebih banyak latihan coding
  \item \textbf{Konsultasi}: Diskusikan dengan dosen atau asisten
  \item \textbf{Study Group}: Bentuk kelompok belajar dengan teman
  \item \textbf{Online Resources}: Manfaatkan tutorial online (Coursera, Udemy, YouTube)
  \item \textbf{Coding Practice}: Gunakan platform seperti HackerRank, LeetCode untuk latihan
\end{enumerate}

\subsection{Sumber Belajar Tambahan}

\textbf{Buku}:
\begin{itemize}
  \item "Effective Java" - Joshua Bloch \cite{ref2}
  \item "Head First Design Patterns" - Freeman \& Robson \cite{ref5}
  \item "Clean Code" - Robert C. Martin \cite{ref3}
  \item "Refactoring" - Martin Fowler \cite{ref6}
  \item "Design Patterns" - Gamma et al. \cite{ref1}
\end{itemize}

\textbf{Online Courses}:
\begin{itemize}
  \item Java Programming and Software Engineering Fundamentals (Coursera)
  \item Object Oriented Programming in Java (Udemy)
  \item Design Patterns in Java (Pluralsight)
\end{itemize}

\textbf{Practice Platforms}:
\begin{itemize}
  \item HackerRank (Java track)
  \item LeetCode (OOP problems)
  \item Codewars
  \item Exercism (Java track)
\end{itemize}


% ============================================================
% RANGKUMAN PERJALANAN PEMBELAJARAN
% ============================================================

\begin{rangkuman}
Selamat! Anda telah menyelesaikan perjalanan pembelajaran Pemrograman Berorientasi Objek. Dari memahami konsep dasar hingga menerapkan design patterns dan best practices, Anda telah membangun fondasi yang kuat dalam pengembangan software modern.

\textbf{Apa yang Telah Anda Pelajari}:
\begin{itemize}
  \item Paradigma OOP dan 4 pilar utama
  \item Class, Object, Constructor, dan Method
  \item Encapsulation dan Information Hiding
  \item Inheritance dan Polymorphism
  \item Abstract Class dan Interface
  \item UML untuk perancangan sistem
  \item Exception Handling
  \item Prinsip SOLID
  \item Design Patterns
  \item Collections dan Generics
  \item File I/O dan Serialization
  \item Unit Testing dan TDD
  \item Refactoring dan Code Quality
\end{itemize}

\textbf{Langkah Selanjutnya}:
Terus praktik, buat proyek nyata, dan jangan berhenti belajar. OOP adalah fondasi, tetapi masih banyak yang bisa dipelajari dalam dunia pengembangan software!
\end{rangkuman}

\end{document}
