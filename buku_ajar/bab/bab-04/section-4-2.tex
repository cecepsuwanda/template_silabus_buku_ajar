\section{Getter/Setter, Validasi, dan Immutability}

\subsection{Getter dan Setter}
Getter dan setter adalah method publik yang mengontrol akses terhadap attribute private. Pola ini memungkinkan validasi dan konsistensi data.

\begin{javacode}[caption={Contoh Getter dan Setter dengan Validasi}]
public class Produk {
    private String kode;
    private String nama;
    private double harga;

    public Produk(String kode, String nama, double harga) {
        this.kode = kode;
        this.nama = nama;
        setHarga(harga);
    }

    public String getKode() {
        return kode;
    }

    public String getNama() {
        return nama;
    }

    public double getHarga() {
        return harga;
    }

    public void setHarga(double harga) {
        if (harga <= 0) {
            throw new IllegalArgumentException("Harga harus positif");
        }
        this.harga = harga;
    }
}
\end{javacode}

\subsection{Immutability}
Object \textit{immutable} tidak dapat diubah setelah dibuat. Pendekatan ini meminimalkan bug dan meningkatkan keamanan data.

\begin{catatan}
Untuk membuat class immutable:
\begin{itemize}
  \item Gunakan \keyword{private final} untuk attributes
  \item Tidak menyediakan setter
  \item Inisialisasi di constructor
  \item Jika menyimpan objek mutable, lakukan defensive copy
\end{itemize}
\end{catatan}

% ============================================================
% AKTIVITAS PEMBELAJARAN
% ============================================================

\begin{aktivitas}
  \item \textbf{Refactoring Encapsulation}: Diberikan class dengan attributes public, ubah menjadi private dan buat getter/setter yang tepat.
  \item \textbf{Validasi Data}: Implementasikan validasi untuk class \class{Mahasiswa} (nilai 0-100, IPK 0.0-4.0).
  \item \textbf{Design Review}: Diskusikan kapan sebaiknya sebuah attribute tidak diberi setter.
  \item \textbf{Studi Kasus}: Buat class \class{BookingHotel} dengan aturan: tanggal check-out harus setelah check-in.
\end{aktivitas}

% ============================================================
% LATIHAN DAN REFLEKSI
% ============================================================

\begin{latihan}
  \item Jelaskan manfaat utama encapsulation dalam pengembangan software.
  \item Apa perbedaan antara attribute \keyword{public} dan \keyword{private}? Berikan contoh.
  \item Buat class \class{KartuMember} dengan attribute nomorMember, nama, poin. Terapkan encapsulation dan validasi poin tidak boleh negatif.
  \item Kapan sebaiknya sebuah class dibuat immutable? Jelaskan dengan contoh.
  \item \textbf{Refleksi}: Bagian apa dari encapsulation yang paling sering Anda lupakan saat menulis kode?
\end{latihan}

% ============================================================
% ASESMEN
% ============================================================

\begin{asesmen}
\textbf{Instrumen Penilaian untuk Sub-CPMK 1.3}

\textbf{A. Pilihan Ganda}
\begin{enumerate}
  \item Tujuan utama encapsulation adalah:
  \begin{enumerate}
    \item Mempercepat eksekusi program
    \item Menyembunyikan detail internal dan mengontrol akses data
    \item Mengurangi jumlah class
    \item Menghapus kebutuhan constructor
  \end{enumerate}
  \item Keyword yang paling tepat untuk attribute yang hanya boleh diakses dalam class:
  \begin{enumerate}
    \item public
    \item protected
    \item private
    \item default
  \end{enumerate}
\end{enumerate}

\textbf{B. Tugas Praktik}
\begin{itemize}
  \item Buat class \class{AkunPengguna} dengan validasi email dan password minimal 8 karakter.
  \item Sertakan getter yang diperlukan dan batasi setter pada data yang tidak boleh diubah.
\end{itemize}

\textbf{Rubrik Penilaian}: Lihat Lampiran A
\end{asesmen}

% ============================================================
% CHECKLIST KOMPETENSI
% ============================================================

\begin{checklist}
  \item Saya dapat menjelaskan konsep encapsulation dan information hiding
  \item Saya dapat menggunakan access modifier dengan tepat
  \item Saya mampu membuat getter/setter dengan validasi
  \item Saya mengetahui kapan menggunakan object immutable
  \item Saya dapat menjaga class invariant dengan encapsulation
\end{checklist}

% ============================================================
% RANGKUMAN
% ============================================================

\begin{rangkuman}
Encapsulation menjaga data tetap aman dan konsisten melalui access modifier dan method pengontrol \cite{ref2}. Getter/setter memungkinkan validasi, sedangkan immutability membantu mencegah perubahan yang tidak diinginkan.

\textbf{Kata Kunci}: \oop{Encapsulation}, \oop{Access Modifier}, \oop{Getter/Setter}, \oop{Immutable}, \oop{Validation}
\end{rangkuman}
