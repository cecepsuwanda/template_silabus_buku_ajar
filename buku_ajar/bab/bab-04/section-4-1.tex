\section{Konsep Encapsulation dan Access Modifier}
\begin{subcpmk}
  \item Sub-CPMK 1.3: Mendemonstrasikan konsep encapsulation dengan access modifier
\end{subcpmk}

\begin{konsep}
\textbf{Encapsulation} adalah prinsip OOP yang menggabungkan data (attributes) dan perilaku (methods) dalam satu class, sekaligus menyembunyikan detail internal dari akses langsung luar \cite{ref2}. Tujuannya adalah menjaga \textit{class invariant} agar data selalu dalam kondisi valid.
\end{konsep}

Dengan encapsulation, kita dapat:
\begin{itemize}
  \item Melindungi data dari perubahan yang tidak sah
  \item Mengontrol akses melalui method publik
  \item Memudahkan perawatan dan perubahan implementasi internal
\end{itemize}

\subsection{Access Modifier di Java}

Java menyediakan access modifier untuk mengatur visibilitas anggota class:

\begin{table}[h]
\centering
\begin{tabular}{|p{3cm}|p{4cm}|p{6cm}|}
\hline
\textbf{Modifier} & \textbf{Akses} & \textbf{Keterangan} \\
\hline
public & Semua class & Dapat diakses dari mana saja \\
\hline
protected & Package + subclass & Dapat diakses dari package yang sama dan subclass \\
\hline
(default) & Package & Tanpa keyword, hanya dalam package yang sama \\
\hline
private & Class sendiri & Hanya dapat diakses di dalam class \\
\hline
\end{tabular}
\caption{Access Modifier di Java}
\end{table}

\subsection{Contoh Encapsulation Sederhana}

\begin{javacode}[caption={Encapsulation pada class RekeningBank}]
public class RekeningBank {
    private String nomorRekening;
    private String namaPemilik;
    private double saldo;

    public RekeningBank(String nomorRekening, String namaPemilik) {
        this.nomorRekening = nomorRekening;
        this.namaPemilik = namaPemilik;
        this.saldo = 0.0;
    }

    public void setor(double jumlah) {
        if (jumlah <= 0) {
            throw new IllegalArgumentException("Jumlah harus positif");
        }
        saldo += jumlah;
    }

    public void tarik(double jumlah) {
        if (jumlah <= 0 || jumlah > saldo) {
            throw new IllegalArgumentException("Saldo tidak mencukupi");
        }
        saldo -= jumlah;
    }

    public double getSaldo() {
        return saldo;
    }
}
\end{javacode}

Contoh di atas menunjukkan bahwa \code{saldo} hanya dapat diubah melalui method \method{setor} dan \method{tarik}, sehingga aturan bisnis dapat dijaga dengan konsisten.
