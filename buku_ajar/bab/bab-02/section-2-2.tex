\section{Paradigma Prosedural vs Berorientasi Objek}

\subsection{Pemrograman Prosedural}

\oop{Pemrograman prosedural} adalah paradigma yang berfokus pada \textbf{prosedur} atau \textbf{fungsi} yang beroperasi pada data.

\textbf{Karakteristik:}
\begin{itemize}
  \item Program terdiri dari serangkaian instruksi/prosedur
  \item Data dan fungsi terpisah
  \item Alur program linear (top-down)
  \item Fokus pada "bagaimana" melakukan sesuatu
\end{itemize}

\begin{contoh}
\textbf{Contoh Kode Prosedural (Pseudo-code):}

\begin{verbatim}
// Data terpisah
struct Mahasiswa {
    string nim;
    string nama;
    double ipk;
}

// Fungsi terpisah
function hitungPredikat(double ipk) {
    if (ipk >= 3.5) return "Cum Laude";
    else if (ipk >= 3.0) return "Sangat Memuaskan";
    else return "Memuaskan";
}

// Penggunaan
Mahasiswa mhs;
mhs.nim = "1234";
mhs.nama = "Budi";
mhs.ipk = 3.7;
string predikat = hitungPredikat(mhs.ipk);
\end{verbatim}

Perhatikan bahwa data (\code{Mahasiswa}) dan fungsi (\code{hitungPredikat}) terpisah.
\end{contoh}

\subsection{Pemrograman Berorientasi Objek}

\oop{Pemrograman Berorientasi Objek (OOP)} adalah paradigma yang berfokus pada \textbf{objek} yang mengenkapsulasi data dan perilaku.

\textbf{Karakteristik:}
\begin{itemize}
  \item Program terdiri dari objek-objek yang berinteraksi
  \item Data dan fungsi digabung dalam satu unit (class)
  \item Fokus pada "apa" yang dimodelkan
  \item Lebih dekat dengan dunia nyata
\end{itemize}

\begin{contoh}
\textbf{Contoh Kode OOP (Java):}

\begin{javacode}[caption={Class Mahasiswa dengan OOP}]
public class Mahasiswa {
    // Data (attributes)
    private String nim;
    private String nama;
    private double ipk;
    
    // Constructor
    public Mahasiswa(String nim, String nama) {
        this.nim = nim;
        this.nama = nama;
        this.ipk = 0.0;
    }
    
    // Behavior (methods)
    public void setIPK(double ipk) {
        this.ipk = ipk;
    }
    
    public String getPredikat() {
        if (ipk >= 3.5) return "Cum Laude";
        else if (ipk >= 3.0) return "Sangat Memuaskan";
        else return "Memuaskan";
    }
}

// Penggunaan
Mahasiswa mhs = new Mahasiswa("1234", "Budi");
mhs.setIPK(3.7);
String predikat = mhs.getPredikat();
\end{javacode}

Data dan fungsi menyatu dalam class \class{Mahasiswa}.
\end{contoh}

\subsection{Perbandingan Prosedural vs OOP}

\begin{table}[h]
\centering
\begin{tabular}{|>{\raggedright\arraybackslash}p{4cm}|>{\raggedright\arraybackslash}p{5cm}|>{\raggedright\arraybackslash}p{5cm}|}
\hline
\textbf{Aspek} & \textbf{Prosedural} & \textbf{OOP} \\
\hline
Fokus & Fungsi/Prosedur & Objek \\
\hline
Struktur Data & Data terpisah dari fungsi & Data dan fungsi dalam class \\
\hline
Akses Data & Global atau parameter & Encapsulation (private/public) \\
\hline
Code Reuse & Function library & Inheritance \\
\hline
Maintainability & Sulit untuk program besar & Lebih mudah dengan modularitas \\
\hline
Real-world Modeling & Abstrak & Lebih natural \\
\hline
\hline
Contoh Bahasa & C, Pascal, FORTRAN & Java, C++, Python \cite{ref4, ref7} \\
\hline
\end{tabular}
\caption{Perbandingan Prosedural vs OOP}
\end{table}
