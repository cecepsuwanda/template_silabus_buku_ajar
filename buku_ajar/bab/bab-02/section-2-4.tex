\section{Keuntungan dan Tantangan OOP}

\subsection{Keuntungan OOP}

\begin{enumerate}
  \item \textbf{Modularitas}: Kode terorganisir dalam class-class yang independen
  \item \textbf{Reusability}: Code reuse melalui inheritance dan composition
  \item \textbf{Maintainability}: Lebih mudah menemukan dan memperbaiki bug
  \item \textbf{Scalability}: Mudah menambah fitur baru tanpa mengubah kode existing
  \item \textbf{Real-world Modeling}: Lebih natural dalam memodelkan dunia nyata
  \item \textbf{Data Security}: Enkapsulasi melindungi data
  \item \textbf{Collaboration}: Tim dapat bekerja pada class yang berbeda secara paralel
\end{enumerate}

\subsection{Tantangan OOP}

\begin{enumerate}
  \item \textbf{Learning Curve}: Membutuhkan pemahaman konsep yang lebih dalam
  \item \textbf{Overhead}: Bisa lebih lambat untuk program sederhana
  \item \textbf{Complexity}: Bisa menjadi terlalu kompleks jika tidak dirancang dengan baik
  \item \textbf{Design Effort}: Membutuhkan perencanaan dan desain yang matang
\end{enumerate}

\begin{catatan}
OOP bukan solusi untuk semua masalah. Untuk program sederhana atau script kecil, pendekatan prosedural mungkin lebih efisien. Gunakan OOP ketika:
\begin{itemize}
  \item Program cukup besar dan kompleks
  \item Membutuhkan code reuse yang tinggi
  \item Banyak developer bekerja pada proyek yang sama
  \item Sistem perlu mudah di-maintain dan di-extend
\end{itemize}
\end{catatan}
