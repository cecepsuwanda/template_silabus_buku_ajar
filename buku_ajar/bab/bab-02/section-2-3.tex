\section{Konsep Dasar OOP}

OOP dibangun di atas empat pilar utama:

\subsection{1. Abstraksi (Abstraction)}

\begin{konsep}
\textbf{Abstraksi} adalah proses menyembunyikan detail implementasi dan hanya menampilkan fungsionalitas kepada pengguna.

Contoh: Ketika menggunakan mobil, Anda hanya perlu tahu cara menggunakan setir, pedal gas, dan rem. Anda tidak perlu tahu detail mesin internal.
\end{konsep}

Dalam OOP, abstraksi dicapai melalui:
\begin{itemize}
  \item Abstract classes
  \item Interfaces
  \item Encapsulation
\end{itemize}

\subsection{2. Enkapsulasi (Encapsulation)}

\begin{konsep}
\textbf{Enkapsulasi} adalah pembungkusan data dan method yang beroperasi pada data tersebut dalam satu unit (class), serta menyembunyikan detail internal dari luar.
\end{konsep}

Manfaat enkapsulasi:
\begin{itemize}
  \item \textbf{Data Hiding}: Melindungi data dari akses tidak sah
  \item \textbf{Modularity}: Kode lebih terorganisir
  \item \textbf{Flexibility}: Mudah mengubah implementasi internal
  \item \textbf{Maintainability}: Lebih mudah dipelihara
\end{itemize}

\subsection{3. Pewarisan (Inheritance)}

\begin{konsep}
\textbf{Inheritance} adalah mekanisme di mana class baru (subclass) dapat mewarisi properties dan methods dari class yang sudah ada (superclass).
\end{konsep}

Manfaat inheritance:
\begin{itemize}
  \item \textbf{Code Reusability}: Tidak perlu menulis ulang kode yang sama
  \item \textbf{Hierarchical Classification}: Organisasi class yang terstruktur
  \item \textbf{Extensibility}: Mudah menambah fitur baru
\end{itemize}

\subsection{4. Polimorfisme (Polymorphism)}

\begin{konsep}
\textbf{Polimorfisme} adalah kemampuan objek untuk mengambil banyak bentuk. Dalam OOP, ini berarti satu interface dapat digunakan untuk tipe data yang berbeda.
\end{konsep}

Jenis polimorfisme:
\begin{itemize}
  \item \textbf{Compile-time Polymorphism}: Method overloading
  \item \textbf{Runtime Polymorphism}: Method overriding
\end{itemize}
