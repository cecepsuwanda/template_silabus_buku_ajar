\documentclass[../main.tex]{subfiles}
\begin{document}

\chapter{Konsep Class dan Object}

\begin{subcpmk}
  \item Sub-CPMK 1.2: Mengidentifikasi class, object, dan attribute dalam studi kasus nyata
  \item Sub-CPMK 1.2: Membuat class sederhana dengan attributes dan methods
  \item Sub-CPMK 1.2: Memahami perbedaan antara class dan object
\end{subcpmk}

% ============================================================
% MATERI POKOK
% ============================================================

\section{Definisi Class dan Object}

\subsection{Apa itu Class?}

\begin{konsep}
\textbf{Class} adalah blueprint atau template untuk membuat objek. Class mendefinisikan attributes (data) dan methods (behavior) yang akan dimiliki oleh objek.

Analogi: Class seperti cetak biru rumah, sedangkan object adalah rumah yang dibangun berdasarkan cetak biru tersebut.
\end{konsep}

Dalam Java, class didefinisikan dengan keyword \keyword{class}:

\begin{javacode}[caption={Struktur Dasar Class}]
public class NamaClass {
    // Attributes (instance variables)
    tipeData namaAttribute;
    
    // Constructor
    public NamaClass() {
        // Inisialisasi
    }
    
    // Methods
    public void namaMethod() {
        // Implementasi
    }
}
\end{javacode}

\subsection{Apa itu Object?}

\begin{konsep}
\textbf{Object} adalah instance (perwujudan) dari class. Object memiliki state (nilai attributes) dan behavior (methods) yang didefinisikan oleh class-nya.
\end{konsep}

Membuat object dalam Java menggunakan keyword \keyword{new}:

\begin{javacode}[caption={Membuat Object}]
NamaClass namaObject = new NamaClass();
\end{javacode}

\subsection{Hubungan Class dan Object}

\begin{table}[h]
\centering
\begin{tabular}{|p{3cm}|p{5cm}|p{5cm}|}
\hline
\textbf{Aspek} & \textbf{Class} & \textbf{Object} \\
\hline
Definisi & Template/Blueprint & Instance dari class \\
\hline
Keberadaan & Logical entity & Physical entity \\
\hline
Memory & Tidak menggunakan memory & Menggunakan memory \\
\hline
Jumlah & Satu class & Banyak object dari satu class \\
\hline
Keyword & \code{class} & \code{new} \\
\hline
\end{tabular}
\end{table}

\section{Attributes (Instance Variables)}

\subsection{Definisi Attributes}

\textbf{Attributes} adalah variabel yang didefinisikan di dalam class untuk menyimpan state/data dari object.

\begin{javacode}[caption={Contoh Attributes}]
public class Mahasiswa {
    // Attributes
    String nim;
    String nama;
    String jurusan;
    int semester;
    double ipk;
}
\end{javacode}

\subsection{Jenis Attributes}

\begin{enumerate}
  \item \textbf{Instance Variables}: Unik untuk setiap object
  \item \textbf{Static Variables}: Dibagi oleh semua object (akan dibahas nanti)
\end{enumerate}

\section{Methods (Behavior)}

\subsection{Definisi Methods}

\textbf{Methods} adalah fungsi yang didefinisikan di dalam class untuk menentukan behavior dari object.

\begin{javacode}[caption={Contoh Methods}]
public class Mahasiswa {
    String nim;
    String nama;
    double ipk;
    
    // Method untuk menampilkan info
    public void tampilkanInfo() {
        System.out.println("NIM: " + nim);
        System.out.println("Nama: " + nama);
        System.out.println("IPK: " + ipk);
    }
    
    // Method untuk mengecek kelulusan
    public boolean lulus() {
        return ipk >= 2.0;
    }
    
    // Method untuk mendapatkan predikat
    public String getPredikat() {
        if (ipk >= 3.5) return "Cum Laude";
        else if (ipk >= 3.0) return "Sangat Memuaskan";
        else if (ipk >= 2.5) return "Memuaskan";
        else return "Cukup";
    }
}
\end{javacode}

\subsection{Jenis Methods}

\begin{enumerate}
  \item \textbf{Accessor Methods (Getter)}: Mengambil nilai attribute
  \item \textbf{Mutator Methods (Setter)}: Mengubah nilai attribute
  \item \textbf{Utility Methods}: Melakukan operasi tertentu
\end{enumerate}

\section{Constructor}

\subsection{Apa itu Constructor?}

\begin{konsep}
\textbf{Constructor} adalah method khusus yang dipanggil saat object dibuat. Constructor digunakan untuk menginisialisasi attributes object.

Ciri-ciri constructor:
\begin{itemize}
  \item Nama sama dengan nama class
  \item Tidak memiliki return type (bahkan bukan \code{void})
  \item Dipanggil otomatis saat object dibuat dengan \keyword{new}
\end{itemize}
\end{konsep}

\begin{javacode}[caption={Contoh Constructor}]
public class Mahasiswa {
    String nim;
    String nama;
    double ipk;
    
    // Constructor tanpa parameter (default constructor)
    public Mahasiswa() {
        nim = "0000";
        nama = "Unknown";
        ipk = 0.0;
    }
    
    // Constructor dengan parameter
    public Mahasiswa(String nim, String nama) {
        this.nim = nim;
        this.nama = nama;
        this.ipk = 0.0;
    }
    
    // Constructor dengan semua parameter
    public Mahasiswa(String nim, String nama, double ipk) {
        this.nim = nim;
        this.nama = nama;
        this.ipk = ipk;
    }
}
\end{javacode}

\subsection{Constructor Overloading}

Class dapat memiliki multiple constructors dengan parameter yang berbeda. Ini disebut \oop{constructor overloading}.

\section{Keyword \texttt{this}}

\subsection{Penggunaan \texttt{this}}

Keyword \keyword{this} merujuk pada object saat ini. Digunakan untuk:

\begin{enumerate}
  \item Membedakan instance variable dengan parameter yang namanya sama
  \item Memanggil constructor lain dalam class yang sama
  \item Mengembalikan instance saat ini
\end{enumerate}

\begin{javacode}[caption={Penggunaan this}]
public class Mahasiswa {
    private String nama;
    private double ipk;
    
    // this untuk membedakan parameter dan instance variable
    public Mahasiswa(String nama, double ipk) {
        this.nama = nama;  // this.nama = instance variable
        this.ipk = ipk;    // nama = parameter
    }
    
    // this untuk memanggil constructor lain
    public Mahasiswa(String nama) {
        this(nama, 0.0);  // Memanggil constructor di atas
    }
    
    // this untuk mengembalikan object saat ini
    public Mahasiswa setNama(String nama) {
        this.nama = nama;
        return this;  // Method chaining
    }
}
\end{javacode}

\section{Object Creation dan Instantiation}

\subsection{Membuat Object}

Proses membuat object disebut \oop{instantiation}. Dalam Java, gunakan keyword \keyword{new}:

\begin{javacode}[caption={Membuat dan Menggunakan Object}]
public class Main {
    public static void main(String[] args) {
        // Membuat object dengan default constructor
        Mahasiswa mhs1 = new Mahasiswa();
        
        // Membuat object dengan parameterized constructor
        Mahasiswa mhs2 = new Mahasiswa("123456", "Budi Santoso");
        
        // Mengakses attributes (jika public)
        mhs2.ipk = 3.75;
        
        // Memanggil methods
        mhs2.tampilkanInfo();
        System.out.println("Predikat: " + mhs2.getPredikat());
        
        // Cek kelulusan
        if (mhs2.lulus()) {
            System.out.println(mhs2.nama + " dinyatakan LULUS");
        }
    }
}
\end{javacode}

\subsection{Memory Allocation}

Ketika object dibuat:
\begin{enumerate}
  \item Memory dialokasikan di \textbf{heap}
  \item Reference variable disimpan di \textbf{stack}
  \item Constructor dipanggil untuk inisialisasi
\end{enumerate}

\begin{catatan}
Jika tidak ada constructor yang didefinisikan, Java otomatis menyediakan \textbf{default constructor} tanpa parameter yang menginisialisasi attributes dengan nilai default (0, null, false).
\end{catatan}

\section{Static vs Instance Members}

\subsection{Instance Members}

\textbf{Instance members} (attributes dan methods) adalah milik object. Setiap object memiliki copy sendiri.

\subsection{Static Members}

\textbf{Static members} adalah milik class, bukan object. Dibagi oleh semua object.

\begin{javacode}[caption={Static vs Instance}]
public class Mahasiswa {
    // Instance variable (setiap object punya copy sendiri)
    private String nama;
    private double ipk;
    
    // Static variable (dibagi semua object)
    private static int jumlahMahasiswa = 0;
    private static String namaUniversitas = "Universitas Lorem Ipsum";
    
    public Mahasiswa(String nama, double ipk) {
        this.nama = nama;
        this.ipk = ipk;
        jumlahMahasiswa++;  // Increment setiap object dibuat
    }
    
    // Instance method
    public void tampilkanInfo() {
        System.out.println("Nama: " + nama);
        System.out.println("IPK: " + ipk);
    }
    
    // Static method
    public static int getJumlahMahasiswa() {
        return jumlahMahasiswa;
    }
    
    public static String getNamaUniversitas() {
        return namaUniversitas;
    }
}

// Penggunaan
public class Main {
    public static void main(String[] args) {
        // Akses static member tanpa object
        System.out.println("Universitas: " + Mahasiswa.getNamaUniversitas());
        
        Mahasiswa mhs1 = new Mahasiswa("Budi", 3.5);
        Mahasiswa mhs2 = new Mahasiswa("Ani", 3.8);
        
        // Akses static member melalui class name
        System.out.println("Total mahasiswa: " + Mahasiswa.getJumlahMahasiswa());
        // Output: Total mahasiswa: 2
    }
}
\end{javacode}

\begin{catatan}
\textbf{Best Practice}:
\begin{itemize}
  \item Akses static members melalui class name, bukan object
  \item Static methods tidak bisa mengakses instance variables secara langsung
  \item Gunakan static untuk utility methods atau constants
\end{itemize}
\end{catatan}

\section{Contoh Lengkap: Class Buku}

\begin{javacode}[caption={Class Buku - Contoh Lengkap}]
public class Buku {
    // Attributes
    private String isbn;
    private String judul;
    private String penulis;
    private int tahunTerbit;
    private double harga;
    private boolean tersedia;
    
    // Static variable
    private static int jumlahBuku = 0;
    
    // Constructor
    public Buku(String isbn, String judul, String penulis, 
                int tahunTerbit, double harga) {
        this.isbn = isbn;
        this.judul = judul;
        this.penulis = penulis;
        this.tahunTerbit = tahunTerbit;
        this.harga = harga;
        this.tersedia = true;
        jumlahBuku++;
    }
    
    // Getter methods
    public String getISBN() {
        return isbn;
    }
    
    public String getJudul() {
        return judul;
    }
    
    public String getPenulis() {
        return penulis;
    }
    
    public int getTahunTerbit() {
        return tahunTerbit;
    }
    
    public double getHarga() {
        return harga;
    }
    
    public boolean isTersedia() {
        return tersedia;
    }
    
    // Setter methods
    public void setHarga(double harga) {
        if (harga > 0) {
            this.harga = harga;
        }
    }
    
    public void setTersedia(boolean tersedia) {
        this.tersedia = tersedia;
    }
    
    // Utility methods
    public void pinjam() {
        if (tersedia) {
            tersedia = false;
            System.out.println("Buku '" + judul + "' berhasil dipinjam");
        } else {
            System.out.println("Buku tidak tersedia");
        }
    }
    
    public void kembalikan() {
        tersedia = true;
        System.out.println("Buku '" + judul + "' berhasil dikembalikan");
    }
    
    public void tampilkanInfo() {
        System.out.println("=== Informasi Buku ===");
        System.out.println("ISBN: " + isbn);
        System.out.println("Judul: " + judul);
        System.out.println("Penulis: " + penulis);
        System.out.println("Tahun: " + tahunTerbit);
        System.out.println("Harga: Rp " + harga);
        System.out.println("Status: " + (tersedia ? "Tersedia" : "Dipinjam"));
    }
    
    // Static method
    public static int getJumlahBuku() {
        return jumlahBuku;
    }
}
\end{javacode}

\begin{javacode}[caption={Penggunaan Class Buku}]
public class TestBuku {
    public static void main(String[] args) {
        // Membuat object buku
        Buku buku1 = new Buku("978-0134685991", 
                              "Effective Java", 
                              "Joshua Bloch", 
                              2018, 
                              450000);
        
        Buku buku2 = new Buku("978-0596009205", 
                              "Head First Java", 
                              "Kathy Sierra", 
                              2005, 
                              350000);
        
        // Menampilkan informasi
        buku1.tampilkanInfo();
        System.out.println();
        
        // Meminjam buku
        buku1.pinjam();
        buku1.pinjam();  // Coba pinjam lagi
        
        // Mengembalikan buku
        buku1.kembalikan();
        
        // Mengubah harga
        buku2.setHarga(300000);
        
        // Menampilkan jumlah buku
        System.out.println("\nTotal buku: " + Buku.getJumlahBuku());
    }
}
\end{javacode}

% ============================================================
% AKTIVITAS PEMBELAJARAN
% ============================================================

\begin{aktivitas}
  \item \textbf{Identifikasi Class}: Dari studi kasus sistem perpustakaan, identifikasi minimal 5 class yang diperlukan beserta attributes dan methods-nya.
  
  \item \textbf{Implementasi Class}: Buat class \class{Mobil} dengan attributes: merk, model, tahun, warna, harga. Tambahkan constructor dan methods yang sesuai.
  
  \item \textbf{Object Interaction}: Buat class \class{Dosen} dan \class{MataKuliah}. Buat program yang mendemonstrasikan interaksi antara object dosen dan mata kuliah.
  
  \item \textbf{Static vs Instance}: Buat class \class{Counter} dengan static variable untuk menghitung jumlah object yang dibuat. Test dengan membuat beberapa object.
  
  \item \textbf{Code Review}: Bertukar kode dengan teman, review class yang dibuat teman Anda. Berikan feedback tentang naming, encapsulation, dan design.
\end{aktivitas}

% ============================================================
% LATIHAN DAN REFLEKSI
% ============================================================

\begin{latihan}
  \item Jelaskan perbedaan antara class dan object! Berikan analogi dari dunia nyata.
  
  \item Apa fungsi constructor? Apa yang terjadi jika tidak mendefinisikan constructor dalam class?
  
  \item Jelaskan perbedaan antara instance variable dan static variable! Kapan sebaiknya menggunakan static variable?
  
  \item Buat class \class{RekeningBank} dengan attributes: nomorRekening, namaPemilik, saldo. Tambahkan methods: setor(), tarik(), cekSaldo(). Implementasikan validasi yang sesuai.
  
  \item Buat class \class{Lingkaran} dengan attribute radius. Tambahkan methods untuk menghitung luas dan keliling. Buat program untuk membuat beberapa object lingkaran dengan radius berbeda.
  
  \item Modifikasi class \class{Mahasiswa} untuk menambahkan array nilai mata kuliah. Tambahkan method untuk menghitung IPK berdasarkan nilai-nilai tersebut.
  
  \item Buat class \class{Produk} untuk sistem e-commerce dengan attributes yang sesuai. Implementasikan method untuk menghitung harga setelah diskon.
  
  \item \textbf{Refleksi}: Apa tantangan terbesar yang Anda hadapi dalam memahami konsep class dan object? Bagaimana Anda mengatasinya?
\end{latihan}

% ============================================================
% ASESMEN
% ============================================================

\begin{asesmen}
\textbf{Instrumen Penilaian untuk Sub-CPMK 1.2}

\textbf{A. Pilihan Ganda}

\begin{enumerate}
  \item Keyword yang digunakan untuk membuat object dalam Java adalah:
  \begin{enumerate}
    \item create
    \item new
    \item object
    \item instance
  \end{enumerate}
  
  \item Manakah pernyataan yang BENAR tentang constructor?
  \begin{enumerate}
    \item Constructor harus memiliki return type
    \item Nama constructor harus berbeda dengan nama class
    \item Constructor dipanggil otomatis saat object dibuat
    \item Satu class hanya boleh memiliki satu constructor
  \end{enumerate}
  
  \item Static variable dalam class:
  \begin{enumerate}
    \item Unik untuk setiap object
    \item Dibagi oleh semua object
    \item Tidak bisa diakses dari luar class
    \item Harus selalu private
  \end{enumerate}
  
  \item Keyword \code{this} digunakan untuk:
  \begin{enumerate}
    \item Merujuk pada superclass
    \item Merujuk pada object saat ini
    \item Membuat object baru
    \item Menghapus object
  \end{enumerate}
\end{enumerate}

\textbf{B. Essay}

\begin{enumerate}
  \item Jelaskan perbedaan antara instance method dan static method! Berikan contoh penggunaan masing-masing.
  
  \item Mengapa encapsulation penting? Jelaskan dengan menggunakan contoh class \class{RekeningBank}.
\end{enumerate}

\textbf{C. Coding Challenge}

\begin{enumerate}
  \item Buat class \class{Karyawan} dengan attributes: nik, nama, jabatan, gajiPokok. Tambahkan:
  \begin{itemize}
    \item Constructor dengan parameter
    \item Getter dan setter methods
    \item Method hitungGajiBersih() yang menghitung gaji setelah potongan pajak 10\%
    \item Method tampilkanInfo()
  \end{itemize}
  
  \item Buat program main yang membuat minimal 3 object karyawan dan menampilkan informasi mereka.
\end{enumerate}

\textbf{Rubrik Penilaian}: Lihat Lampiran A
\end{asesmen}

% ============================================================
% CHECKLIST KOMPETENSI
% ============================================================

\begin{checklist}
  \item Saya dapat menjelaskan perbedaan antara class dan object
  \item Saya dapat membuat class dengan attributes dan methods
  \item Saya dapat membuat constructor dengan dan tanpa parameter
  \item Saya dapat membuat dan menggunakan object dari class
  \item Saya memahami penggunaan keyword \code{this}
  \item Saya dapat membedakan instance variable dan static variable
  \item Saya dapat mengidentifikasi class dan object dari studi kasus nyata
  \item Saya dapat mengimplementasikan getter dan setter methods
\end{checklist}

% ============================================================
% RANGKUMAN
% ============================================================

\begin{rangkuman}
Bab ini membahas konsep fundamental OOP: Class dan Object. Class adalah blueprint untuk membuat object, sedangkan object adalah instance dari class.

\textbf{Poin Kunci:}
\begin{itemize}
  \item Class terdiri dari attributes (data) dan methods (behavior)
  \item Object dibuat menggunakan keyword \keyword{new}
  \item Constructor adalah method khusus untuk inisialisasi object
  \item Keyword \keyword{this} merujuk pada object saat ini
  \item Instance members milik object, static members milik class
  \item Encapsulation dicapai dengan access modifiers dan getter/setter
\end{itemize}

\textbf{Kata Kunci}: \oop{Class}, \oop{Object}, \oop{Attribute}, \oop{Method}, \oop{Constructor}, \oop{this}, \oop{static}, \oop{instance}, \oop{instantiation}
\end{rangkuman}

\end{document}
