\section{Unit Testing dan Siklus TDD}
\begin{subcpmk}
  \item Sub-CPMK 4.1: Menganalisis code smell dan melakukan refactoring
\end{subcpmk}

\subsection{Konsep Unit Testing}
Unit testing memastikan setiap method bekerja sesuai harapan. Pengujian yang baik mempercepat deteksi bug.

\subsection{Test-Driven Development (TDD)}
Siklus TDD: \textbf{Red} (tulis test gagal), \textbf{Green} (buat test lulus), \textbf{Refactor} (perbaiki struktur).

\textbf{Refactoring} adalah proses mengubah sistem perangkat lunak sedemikian rupa sehingga tidak mengubah perilaku eksternal kode namun memperbaiki struktur internalnya \cite{ref6}.

\begin{javacode}[caption={Contoh Unit Test JUnit}]
import static org.junit.jupiter.api.Assertions.assertEquals;
import org.junit.jupiter.api.Test;

class KalkulatorTest {
    @Test
    void testTambah() {
        Kalkulator k = new Kalkulator();
        assertEquals(5, k.tambah(2, 3));
    }
}
\end{javacode}
