\section{Code Quality dan Refactoring}

\subsection{Code Smell yang Umum}
Beberapa contoh code smell:
\begin{itemize}
  \item Method terlalu panjang
  \item Duplikasi kode
  \item Nama variabel tidak jelas
  \item Kelas melakukan terlalu banyak tugas
\end{itemize}

\subsection{Teknik Refactoring}
Refactoring adalah proses memperbaiki struktur kode tanpa mengubah perilaku.
\begin{itemize}
  \item Extract Method
  \item Rename Variable
  \item Replace Magic Number with Constant
  \item Move Method
\end{itemize}

% ============================================================
% AKTIVITAS PEMBELAJARAN
% ============================================================

\begin{aktivitas}
  \item \textbf{Unit Test}: Buat unit test untuk class \class{RekeningBank}.
  \item \textbf{TDD}: Terapkan siklus Red-Green-Refactor pada fitur login sederhana.
  \item \textbf{Refactoring}: Perbaiki class dengan method yang terlalu panjang.
  \item \textbf{Diskusi}: Identifikasi code smell pada proyek mini.
\end{aktivitas}

% ============================================================
% LATIHAN DAN REFLEKSI
% ============================================================

\begin{latihan}
  \item Jelaskan manfaat unit testing bagi tim pengembang.
  \item Buat test case untuk method \method{hitungDiskon} pada class \class{Produk}.
  \item Sebutkan minimal 3 contoh refactoring dan kapan digunakan.
  \item Bagaimana TDD membantu desain yang lebih baik?
  \item \textbf{Refleksi}: Apa code smell yang paling sering Anda temui?
\end{latihan}

% ============================================================
% ASESMEN
% ============================================================

\begin{asesmen}
\textbf{Instrumen Penilaian untuk Sub-CPMK 4.1}

\textbf{A. Pilihan Ganda}
\begin{enumerate}
  \item Siklus TDD yang benar adalah:
  \begin{enumerate}
    \item Green-Red-Refactor
    \item Red-Green-Refactor
    \item Refactor-Red-Green
    \item Red-Refactor-Green
  \end{enumerate}
  \item Refactoring bertujuan untuk:
  \begin{enumerate}
    \item Mengubah perilaku program
    \item Memperbaiki struktur tanpa mengubah perilaku
    \item Menambah fitur baru
    \item Menghapus semua test
  \end{enumerate}
\end{enumerate}

\textbf{B. Tugas Praktik}
\begin{itemize}
  \item Lakukan refactoring pada class \class{Order} agar mengikuti SRP dan tambahkan unit test.
\end{itemize}

\textbf{Rubrik Penilaian}: Lihat Lampiran A
\end{asesmen}

% ============================================================
% CHECKLIST KOMPETENSI
% ============================================================

\begin{checklist}
  \item Saya memahami konsep unit testing
  \item Saya memahami siklus TDD
  \item Saya dapat mengidentifikasi code smell
  \item Saya dapat melakukan refactoring dasar
  \item Saya dapat menulis test untuk method penting
\end{checklist}

% ============================================================
% RANGKUMAN
% ============================================================

\begin{rangkuman}
Unit testing dan TDD meningkatkan kualitas software. Refactoring membantu menjaga struktur kode tetap bersih dan mudah dipelihara, sebagaimana didokumentasikan dalam katalog refactoring \cite{ref6}.

\textbf{Kata Kunci}: \oop{Unit Testing}, \oop{TDD}, \oop{Refactoring}, \oop{Code Smell}, \oop{JUnit}
\end{rangkuman}
