\section{Pengenalan Design Patterns dan Creational Patterns}
\begin{subcpmk}
  \item Sub-CPMK 4.2: Mengimplementasikan minimal 3 design patterns (Singleton, Factory, Observer)
\end{subcpmk}

\begin{konsep}
\textbf{Design pattern} adalah solusi umum untuk masalah desain yang sering muncul. Pattern membantu menghasilkan kode yang terstruktur dan mudah dipelihara.
\end{konsep}

\subsection{Singleton Pattern}
Menjamin hanya ada satu instance dari class.

\begin{javacode}[caption={Singleton Pattern}]
public class DatabaseConnection {
    private static DatabaseConnection instance;

    private DatabaseConnection() {}

    public static DatabaseConnection getInstance() {
        if (instance == null) {
            instance = new DatabaseConnection();
        }
        return instance;
    }
}
\end{javacode}

\subsection{Factory Pattern}
Menyediakan cara membuat objek tanpa mengekspos logika instansiasi.

\begin{javacode}[caption={Factory Pattern}]
interface Kendaraan {
    void jalan();
}

class Mobil implements Kendaraan {
    public void jalan() {
        System.out.println("Mobil berjalan");
    }
}

class Motor implements Kendaraan {
    public void jalan() {
        System.out.println("Motor berjalan");
    }
}

class KendaraanFactory {
    public static Kendaraan buat(String jenis) {
        if ("mobil".equalsIgnoreCase(jenis)) return new Mobil();
        if ("motor".equalsIgnoreCase(jenis)) return new Motor();
        return null;
    }
}
\end{javacode}
