\section{Design Patterns: Creational Patterns}
\begin{subcpmk}
  \item Sub-CPMK 4.2: Mengimplementasikan minimal 3 design patterns (Singleton, Factory, Observer)
\end{subcpmk}

\begin{konsep}
\textbf{Design Pattern} adalah solusi umum untuk masalah yang sering muncul dalam konteks desain perangkat lunak \cite{ref1}. Pattern bukanlah desain akhir yang dapat langsung diubah menjadi kode, melainkan template tentang cara menyelesaikan masalah.
\end{konsep}

\subsection{Singleton Pattern}
Menjamin hanya ada satu instance dari class.

\begin{javacode}[caption={Singleton Pattern}]
public class DatabaseConnection {
    private static DatabaseConnection instance;

    private DatabaseConnection() {}

    public static DatabaseConnection getInstance() {
        if (instance == null) {
            instance = new DatabaseConnection();
        }
        return instance;
    }
}
\end{javacode}

\begin{figure}[h]
\centering
\begin{tikzpicture}
  \node (singleton) [class, minimum width=4.5cm] {
    \begin{tabular}{c}
      \textbf{SingletonClass} \\
      \hline
      - instance: SingletonClass \\
      \hline
      + getInstance(): SingletonClass \\
      - SingletonClass()
    \end{tabular}
  };
  \draw [arrow, out=30, in=330, looseness=3] (singleton.east) to node[right] {creates} (singleton.east);
\end{tikzpicture}
\caption{Struktur Singleton Pattern}
\end{figure}

\subsection{Factory Pattern}
Menyediakan cara membuat objek tanpa mengekspos logika instansiasi.

\begin{javacode}[caption={Factory Pattern}]
interface Kendaraan {
    void jalan();
}

class Mobil implements Kendaraan {
    public void jalan() {
        System.out.println("Mobil berjalan");
    }
}

class Motor implements Kendaraan {
    public void jalan() {
        System.out.println("Motor berjalan");
    }
}

class KendaraanFactory {
    public static Kendaraan buat(String jenis) {
        if ("mobil".equalsIgnoreCase(jenis)) return new Mobil();
        if ("motor".equalsIgnoreCase(jenis)) return new Motor();
        return null;
    }
}
\end{javacode}

\begin{figure}[h]
\centering
\begin{tikzpicture}[node distance=1.5cm]
  \node (factory) [class] {Factory};
  \node (interface) [class, fill=gray!20, right=of factory] {\textit{Product}};
  \node (concrete) [class, below=of interface] {ConcreteProduct};
  
  \draw [arrow] (factory) -- (interface) node[midway, above] {creates};
  \draw [arrow] (concrete) -- (interface);
\end{tikzpicture}
\caption{Struktur Factory Pattern}
\end{figure}
