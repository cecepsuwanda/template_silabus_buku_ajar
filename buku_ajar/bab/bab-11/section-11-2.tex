\section{Behavioral Pattern: Observer}

\subsection{Observer Pattern}
Observer digunakan ketika satu objek (subject) perlu memberi tahu banyak objek lain (observer) saat ada perubahan.

\begin{javacode}[caption={Observer Pattern Sederhana}]
interface Observer {
    void update(String pesan);
}

class Subscriber implements Observer {
    private String nama;

    public Subscriber(String nama) {
        this.nama = nama;
    }

    public void update(String pesan) {
        System.out.println(nama + " menerima: " + pesan);
    }
}
\end{javacode}

\subsection{Kapan Menggunakan Design Pattern}
Gunakan pattern saat:
\begin{itemize}
  \item Solusi berulang muncul di banyak bagian aplikasi
  \item Dibutuhkan struktur yang konsisten
  \item Tim perlu pola komunikasi yang jelas
\end{itemize}

% ============================================================
% AKTIVITAS PEMBELAJARAN
% ============================================================

\begin{aktivitas}
  \item \textbf{Implementasi Singleton}: Buat class \class{Logger} sebagai singleton.
  \item \textbf{Factory}: Buat factory untuk membuat objek pembayaran.
  \item \textbf{Observer}: Buat sistem notifikasi berita dengan subject dan observer.
  \item \textbf{Diskusi}: Identifikasi pattern pada framework Java (misalnya Spring).
\end{aktivitas}

% ============================================================
% LATIHAN DAN REFLEKSI
% ============================================================

\begin{latihan}
  \item Jelaskan perbedaan Factory dan Singleton.
  \item Buat contoh kasus yang cocok menggunakan Observer.
  \item Apa resiko jika Singleton digunakan berlebihan?
  \item Tambahkan pattern Strategy pada studi kasus diskon.
  \item \textbf{Refleksi}: Pattern mana yang paling membantu desain Anda?
\end{latihan}

% ============================================================
% ASESMEN
% ============================================================

\begin{asesmen}
\textbf{Instrumen Penilaian untuk Sub-CPMK 4.2}

\textbf{A. Pilihan Ganda}
\begin{enumerate}
  \item Pattern yang menjamin hanya satu instance adalah:
  \begin{enumerate}
    \item Factory
    \item Singleton
    \item Observer
    \item Strategy
  \end{enumerate}
  \item Observer pattern cocok digunakan untuk:
  \begin{enumerate}
    \item Membuat banyak objek
    \item Mengelola notifikasi perubahan
    \item Menghitung data
    \item Menyimpan konfigurasi
  \end{enumerate}
\end{enumerate}

\textbf{B. Tugas Praktik}
\begin{itemize}
  \item Implementasikan Singleton, Factory, dan Observer pada satu studi kasus sederhana.
\end{itemize}

\textbf{Rubrik Penilaian}: Lihat Lampiran A
\end{asesmen}

% ============================================================
% CHECKLIST KOMPETENSI
% ============================================================

\begin{checklist}
  \item Saya memahami konsep design patterns
  \item Saya dapat menerapkan Singleton
  \item Saya dapat menerapkan Factory
  \item Saya dapat menerapkan Observer
  \item Saya dapat memilih pattern sesuai kebutuhan
\end{checklist}

% ============================================================
% RANGKUMAN
% ============================================================

\begin{rangkuman}
Design patterns menyediakan solusi teruji untuk masalah desain umum \cite{ref1, ref5}. Singleton, Factory, dan Observer adalah pola dasar yang sering digunakan di aplikasi nyata.

\textbf{Kata Kunci}: \oop{Design Pattern}, \oop{Singleton}, \oop{Factory}, \oop{Observer}, \oop{Strategy}
\end{rangkuman}
