\chapter*{Lampiran}
\addcontentsline{toc}{chapter}{Lampiran}

% ============================================================
% Lampiran A: Rubrik Penilaian
% ============================================================
\section*{Lampiran A: Rubrik Penilaian Tugas Praktik}

\begin{table}[h]
\centering
\small
\begin{tabular}{|>{\raggedright\arraybackslash}p{3cm}|>{\raggedright\arraybackslash}p{3.5cm}|>{\raggedright\arraybackslash}p{3.5cm}|>{\raggedright\arraybackslash}p{3.5cm}|}
\hline
\textbf{Kriteria} & \textbf{Sangat Baik} & \textbf{Baik} & \textbf{Perlu Perbaikan} \\
\hline
Desain Class & Struktur jelas, relasi tepat & Struktur cukup jelas & Struktur belum tepat \\
\hline
Encapsulation & Semua data terlindungi & Sebagian data terlindungi & Banyak data terbuka \\
\hline
Implementasi OOP & Inheritance dan polymorphism tepat & Ada kekurangan kecil & Banyak kesalahan konsep \\
\hline
Code Quality & Naming rapi, tidak ada duplikasi & Ada sedikit duplikasi & Banyak duplikasi dan nama tidak jelas \\
\hline
Testing & Test lengkap dan relevan & Test sebagian & Tidak ada test \\
\hline
\end{tabular}
\caption{Rubrik Penilaian Tugas Praktik}
\end{table}

% ============================================================
% Lampiran B: Contoh Template Laporan
% ============================================================
\section*{Lampiran B: Contoh Template Laporan Tugas}
\begin{enumerate}
  \item Judul dan Identitas Mahasiswa
  \item Deskripsi Masalah
  \item Desain Class dan UML
  \item Implementasi (cuplikan kode penting)
  \item Hasil Pengujian
  \item Refleksi dan Kesimpulan
\end{enumerate}

% ============================================================
% Lampiran C: Glosarium
% ============================================================
\section*{Lampiran C: Glosarium Istilah OOP}
\begin{itemize}
  \item \textbf{Class}: Blueprint untuk membuat objek
  \item \textbf{Object}: Instance dari class
  \item \textbf{Encapsulation}: Penyembunyian data melalui akses terkontrol
  \item \textbf{Inheritance}: Pewarisan atribut dan method
  \item \textbf{Polymorphism}: Satu interface untuk banyak bentuk
  \item \textbf{Abstraction}: Menyembunyikan detail implementasi
  \item \textbf{Interface}: Kontrak perilaku tanpa implementasi
  \item \textbf{Refactoring}: Perbaikan struktur kode tanpa mengubah perilaku
\end{itemize}

% ============================================================
% Lampiran D: Referensi Tambahan
% ============================================================
\section*{Lampiran D: Referensi Tambahan}
\begin{itemize}
  \item Oracle Java Documentation: \url{https://docs.oracle.com/javase/}
  \item JUnit 5 User Guide: \url{https://junit.org/junit5/docs/current/user-guide/}
  \item Refactoring Catalog: \url{https://refactoring.com/catalog/}
\end{itemize}
