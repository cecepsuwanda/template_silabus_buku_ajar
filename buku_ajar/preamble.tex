% Preamble bersama untuk Buku Ajar OBE Modular
% Jangan sertakan \begin{document} di sini

\usepackage[utf8]{inputenc}
\usepackage[T1]{fontenc}
\usepackage[indonesian]{babel}
\usepackage{geometry}
\geometry{a4paper, margin=2.5cm, headheight=15pt}

\usepackage{graphicx}
\usepackage{booktabs}
\usepackage{longtable}
\usepackage{array}
\usepackage{enumitem}
\usepackage{hyperref}
\usepackage{subfiles}
\usepackage{fancyhdr}

\hypersetup{
  colorlinks=true,
  linkcolor=blue,
  urlcolor=blue,
  citecolor=blue
}

% Pengaturan header/footer (opsional, bisa disesuaikan)
\pagestyle{fancy}
\fancyhf{}
\fancyhead[LE,RO]{\thepage}
\fancyhead[RE]{\leftmark}
\fancyhead[LO]{\rightmark}
\renewcommand{\headrulewidth}{0.4pt}

% Lingkungan untuk blok Sub-CPMK / capaian per bab (OBE)
\newenvironment{subcpmk}{
  \noindent\textbf{Sub-CPMK yang dicakup:}
  \begin{itemize}[leftmargin=*]
}{
  \end{itemize}
}

% Lingkungan untuk latihan (OBE)
\newenvironment{latihan}{
  \par\noindent\textbf{Latihan}
  \begin{enumerate}[leftmargin=*]
}{
  \end{enumerate}
}
